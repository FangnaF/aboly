% Lengths.
\newlength{\lyszbody}
\newlength{\lyszchaptertitle}
\newlength{\lyszsectiontitle}
\newlength{\lyszsubsectiontitle}
\newlength{\lyszpageno}

\setlength{\lyszbody}{12pt}
\setlength{\lyszchaptertitle}{16pt}
\setlength{\lyszsectiontitle}{14pt}
\setlength{\lyszsubsectiontitle}{13pt}
\setlength{\lyszpageno}{10.5pt}


% Page layout.
\usepackage[
  left=3.2cm,
  right=3.2cm,
  top=3.5cm,
  bottom=3.5cm,
]{geometry}


% Packages.
\usepackage{calc}
\usepackage{enumitem}
\usepackage{etoolbox}
%\usepackage{xstring}
\usepackage{graphicx}
\usepackage{indentfirst}
\usepackage[svgnames]{xcolor}
\usepackage[normalem]{ulem}

\usepackage[pagecolor=white, nopagecolor=white]{pagecolor}
\usepackage{afterpage}
\usepackage[scale=1.5]{ccicons}

\usepackage{tikz}
\usetikzlibrary{mindmap}

\usepackage{tcolorbox}
\tcbuselibrary{skins, breakable}


% Fonts.

% English, numerics, symbols.
\usepackage{pifont}
\usepackage[quiet]{fontspec}
\defaultfontfeatures{Mapping=tex-text}
\setmainfont[Ligatures=NoCommon]{Linux Libertine O}
\setsansfont[Ligatures=NoCommon, Scale=MatchLowercase]{Linux Libertine O}

\font\pagenofont = "Linux Libertine O" at \lyszpageno  % for page numbers

% Pinyin. The second tone marks of "Linux Libertine O" is upside-down for the Chinese.
% Besides, it seems conventional to use a Sans Serif font for pinyin. Also good for labeling.
% The cover page's pinyin won't be typeset in this style.
\newfontfamily{\ftpinyin}[Scale=0.86]{Source Han Sans SC}
\newcommand{\lypy}[1]{\textnormal{\ftpinyin #1}}

% 八卦。
\newcommand\lybaguasym[1]{{\CJKfontspec{DejaVu Sans}#1}}

% Chinese.
%
% 方正字体授权(2016年2月起正式施行): http://www.foundertype.com/index/release_info.html
% \lyq{
%  计算机字库的使用方式主要有以下三种:
%    1. 内部使用:指个人或单位在其内部使用的终端设备上安装并使用字库的行为。该使用行为包括在屏幕上显示和临时从打印机上输出两种。
%    2. 内置使用:指将字库文件整体或部分、直接或格式转换后,以嵌入的方式应用到网站、计算机程序或带有可视化显示功能的电子产品中的行为。
%    3. 商业发布:指以直接营利或者间接营利为目的,将字体作为视觉设计要素,进行复制、发行、展览、放映、信息网络传播、广播等使用字体的行为。
%  }
% \lyq{上述三种方式中 “商业发布”行为是字体应用最重要的一种方式。方正公司仅针对商业发布内容的最终所有者提供授权。}
% \lyq{免费字体:包括四种字体:方正黑体、方正书宋、方正仿宋、方正楷体。针对“商业发布”这种使用方式免费。}
%
% 根据我的理解,由于“内置使用”弱于“商业发布”,这表示对“内置使用”这种方式的开源项目也免费。
% 客服解释,这4个字体虽然商业使用免费,但字体文件仍需付费下载。GBK版每个字体5元。
%
% 另外:
% 1. 既然我已经付费下载,获得它们的合法使用权;
% 2. “商业发布”中,又允许“复制、发行”等行为。
% 所以是不是可以说,我有权把这4个字体包含在自己的项目里,以任意方式(包括开源)发布,使所有从我这里获得这4个字体的人,
% 也递推具有了“商业发布”的合法使用权,而不用再像我那样付费下载了?请熟悉这一块法律的朋友指教。
\usepackage{xeCJK}
\usepackage{zhnumber}
\setCJKmainfont[BoldFont={方正黑体_GBK}, ItalicFont={方正楷体_GBK}]{方正书宋_GBK}
\setCJKfamilyfont{fttitle}{方正黑体_GBK}
\setCJKfamilyfont{ftfangsong}[Scale=1.03]{方正仿宋_GBK}
\setCJKfamilyfont{ftkaiti}[Scale=1.03]{方正楷体_GBK}

% SURSONG.TTF, 宋体-方正超大字符集, distributed with Windows XP Simplified Chinese version, Office 2003, etc.
\setCJKfamilyfont{ftfallback}{Simsun (Founder Extended)}


% Header & footer.
\usepackage{fancyhdr}
\pagestyle{fancy}

\fancyhf{}
\newcommand{\lyheaderimg}{}  % placeholder
\chead{\raisebox{-4pt}{\includegraphics{img/\lyheaderimg}}}  % use outlined graphics
\cfoot{\textcolor{DarkSlateGray}{\pagenofont - \thepage\ -}}

\fancypagestyle{plain}{%
\renewcommand{\headrule}{}
\fancyhf{}
\cfoot{\textcolor{DarkSlateGray}{\pagenofont - \thepage\ -}}
}


% Header rule.
\newcommand{\lymakeheadrule}{%
  \makebox[0pt][l]{\rule[0.55\baselineskip]{\headwidth}{0.4pt}}%
  \rule[0.7\baselineskip]{\headwidth}{1pt}}

\makeatletter
\renewcommand{\headrule}{%
  {\if@fancyplain\let\headrulewidth\plainheadrulewidth\fi%
  \lymakeheadrule}}
\makeatother

\setlength{\headheight}{34pt}  % suppress fancyhdr warnings

% Remove space before chapter title.
\makeatletter
\patchcmd{\@makechapterhead}{50}{0}{}{}
\patchcmd{\@makeschapterhead}{50}{0}{}{}
\makeatother

% titlesec.
\usepackage{titlesec}
\titleformat{\chapter}[block]
{\fontsize{\lyszchaptertitle}{\lyszchaptertitle * \real{1.2}}\selectfont\CJKfamily{fttitle}\filcenter}
{第\zhnum{chapter}篇}
{0.5em}{}

\titleformat{\section}[block]
{\fontsize{\lyszsectiontitle}{\lyszsectiontitle * \real{1.2}}\selectfont\CJKfamily{fttitle}}
{\thesection}
{1em}{}

\titleformat{\subsection}[block]
{\fontsize{\lyszsubsectiontitle}{\lyszsectiontitle * \real{1.2}}\selectfont\CJKfamily{fttitle}}
{\thesubsection}
{1em}{}

\titlespacing*{\chapter}{0pt}{0pt}{30pt}

% titletoc.
\usepackage{titletoc}
\titlecontents{chapter}
  [0em]
  {\bfseries}
  {第\zhnumber{\thecontentslabel}篇\hspace{0.5em}}
  {}
  {\dotfill\textrm{\contentspage[{\bfseries\thecontentspage}]}}
\titlecontents{section}
  [2em]
  {\bfseries}
  {}
  {}
  {\dotfill\textrm{\contentspage[{\bfseries\thecontentspage}]}}


%%%%%%%%%%%%%%%%%%%%%%%%%%%%%%%%%%%%%%%%%%%%%%%%%%%%%%%%%%%%
% Other tweaks.
%%%%%%%%%%%%%%%%%%%%%%%%%%%%%%%%%%%%%%%%%%%%%%%%%%%%%%%%%%%%

\setlength{\parindent}{\lyszbody * 2}
\renewcommand{\baselinestretch}{1.4}
\renewcommand{\labelitemii}{$\triangleright$}


% Chinese names.
\renewcommand{\contentsname}{目\qquad 录}
\renewcommand{\listfigurename}{插\quad 图\quad 目\quad 录}
\renewcommand{\listtablename}{表\quad 格\quad 目\quad 录}
\renewcommand{\figurename}{图}
\renewcommand{\tablename}{表}


% Underlines.
% color = gray, depth = 0.32em, thickness = 0.54pt
\newcommand{\lybasiculine}{\bgroup\markoverwith{\textcolor{gray}{\rule[-0.32em]{1pt}{0.54pt}}}\ULon}

% Don't join up with neighbors.
\newcommand{\lyuline}[1]{%
  \nobreak\hspace{0.1em}%
  \lybasiculine{\kern-0.1em\hbox{#1\kern-0.1em}}%
  \nobreak\hspace{0.1em}}

\newcommand{\lydashuline}{\bgroup\markoverwith{%
  \kern1.2pt\textcolor{gray}{\rule[-0.32em]{2.6pt}{0.54pt}}\kern1.2pt}%
  \ULon}

% Not used for now.
%\newcommand{\lydbuline}{%
%  \bgroup\markoverwith{%
%  \textcolor{gray}{\rule[-0.25em]{1pt}{0.3pt}}%
%  \llap{\textcolor{gray}{\rule[-0.35em]{1pt}{0.3pt}}}%
%  }\ULon}


% Keywords, quotations, etc.
\newcommand{\lybit}[1]{\textbf{#1}}  % highlight to be explained
\newcommand{\lykw}[1]{{\CJKfamily{ftkaiti}#1}}  % stressed keyword
\newcommand{\lyq}[1]{{\CJKfamily{ftfangsong}#1}}  % inline quote
\newcommand{\lychar}[1]{\lyuline{#1}} % character in text
\newcommand{\lytopics}[1]{}  % processed by external script; use ,as separator
\newcommand{\lycenterpar}[1]{\noindent\begin{center}#1\end{center}}

\newcommand{\lycharname}[1]{\textbf{#1}}
\newcommand{\lybook}[1]{\textbf{#1}}
\newcommand{\lybooke}[1]{\textbf{#1}}


% Labels and links.
\makeatletter

% Create a hypertarget with PDF bookmark location raised by one line.
\newcommand{\lylabel}[1]{\Hy@raisedlink{\hypertarget{#1}{}}}

% Links.
\newcommand{\lycharlink}[2]{\hyperlink{#1}{\lyuline{#2}}}  % character

% Double-underline is visually distracting. worthwhile or not
% to highlight links to blobs? Revert to old style.
%\newcommand{\lylink}[2]{\hyperlink{#1}{%
%  \IfDecimal{#1}{\lydbuline{#2}}%  % blob
%    {\lyuline{#2}}}}               % other
\newcommand{\lylink}[2]{\hyperlink{#1}{{\lyuline{#2}}}}

\newcommand{\lyref}[1]{\hyperlink{#1}{\tcbox[nobeforeafter, size=fbox, box align=base, colback=white, colframe=black!50, boxrule=0.36mm]{#1}}}  % \lyblob

\newcommand{\lyurl}[2]{\href{#1}{\lydashuline{#2}}}  % Web page

% Used by \lyblob. Create a raised hypertarget and add a PDF bookmark.
\newcommand{\lyblobmark}[2][1]{\Hy@raisedlink{\hypertarget{#2}{}\pdfbookmark[#1]{#2}{#2}}}

% Create a PDF bookmark directly; the default level is section.
\def\lymunch#1{}
\newcommand{\lypdfbookmark}[4][1]{%
  #2#3{\phantomsection\Hy@writebookmark{}{#4}{\@currentHref}{#1}{toc}#4}}

\makeatother


% Environments.

% Compact version of itemize and enumerate.
\newenvironment{lyitemize}%
{\begin{itemize}[leftmargin=*, topsep=0.32em]%
  \setlength{\itemsep}{2pt}%
  \setlength{\parskip}{2pt}}%
{\end{itemize}}

\newenvironment{lyenumerate}%
{\begin{enumerate}[leftmargin=*, topsep=0.32em]%
  \setlength{\itemsep}{2pt}%
  \setlength{\parskip}{2pt}}%
{\end{enumerate}}

% Chinese poems quotation.
\newenvironment{lyquotepoem}%
{\begin{quotation}\CJKfamily{ftfangsong}}%
{\end{quotation}}

% English poems quotation.
\newenvironment{lyquotepoeme}%
{\begin{quotation}}%
{\end{quotation}}

% Paragraphs quotation.
\newenvironment{lyquotepara}%
{\begin{quotation}\CJKfamily{ftfangsong}}%
{\end{quotation}}

\newenvironment{lysignature}%
{\vspace{2em}\begin{flushright}\itshape}%
{\end{flushright}}


% Blobs.
\colorlet{lytextbackground}{black!10}
\newtcolorbox[auto counter, number within=chapter]{lybaseblob}[1]{
  breakable,
  skin=enhanced jigsaw,
  colback=white, colframe=lytextbackground, colbacktitle=lytextbackground, coltitle=black,
  boxrule=0.5mm,
  segmentation style={solid, line width=0.4mm},
  phantom={\lyblobmark{\thetcbcounter}},
  title=#1,
  overlay unbroken and first ={
    \node[anchor=north east, inner sep=0pt, outer sep=0pt] at ([xshift=-\marginparsep]frame.north west) {\tcbox[boxsep=0pt, left=1.8mm, right=1.8mm, top=1.8mm, bottom=1.8mm, colback=white, colframe=black!25]{\thetcbcounter}};
  }
}

% Normal style.
\newcommand{\lyblob}[3]{%
  \begin{lybaseblob}{#1}%
    \begin{lyitemize}
    #2
    \end{lyitemize}
    \ifblank{#3}{}{\tcblower #3}%
  \end{lybaseblob}%
}

% Free style.
\newcommand{\lybloba}[3]{%
  \begin{lybaseblob}{#1}%
    #2%
    \ifblank{#2}{\typeout{Empty upper part}}{}%
    \ifblank{#3}{}{\tcblower #3}%
  \end{lybaseblob}%
}


% Version info.
\providecommand{\lycommitno}{0}
\providecommand\buildnoaboly{0}
\providecommand{\buildnoly}{0}
\newcommand{\projectversionnumber}{v0.3.0}
\newcommand{\versioninfoaboly}{\projectversionnumber+build.\buildnoaboly.commit.\lycommitno}
\newcommand{\versioninfoly}{\projectversionnumber+build.\buildnoly.commit.\lycommitno}
