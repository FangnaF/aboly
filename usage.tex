\lylabel{usage}
\phantomsection
\chapter*{使\quad 用\quad 说\quad 明}
\addcontentsline{toc}{chapter}{使用说明}
%%%%%%%%%%%%%%%%%%%%%%%%%%%%%%%%%%%%%%%%%%%%%%%%%%%%%%%%%%%%

写一本书出来,总希望它讨人喜欢。这里讲一讲它为什么是现在这个样子,并且在我的构想中,它还会变成什么样。


\lypdfbookmark\section*{命名}
%%%%%%%%%%%%%%%%%%%%%%%%%%%%%%%%%%%%%%%%%%%%%%%%%%%%%%%%%%%%

“一瓶论语”这个名字,主观上想说,两千多年来,《论语》的研究心得早已汇聚成汪洋大海,这本小书只是我在浅岸边凭着喜好汲取的点点滴滴,肯定有很多“一瓶子不满,半瓶子晃荡”的毛病。你不妨以此为阶,形成对《论语》完整独立的理解,装满让自己受用不尽的一瓶。

内容上,“一瓶”可以看成“译评”的谐音,本书的注解大致分成译和评两种性质。

说“译”并不准确,它没有像很多现代注本那样,把原文全部翻译出来。我一直觉得,无论文言文还是外语,只要不是刚入门或者非常难的结构,大段的翻译就很碍眼。它占据了太多篇幅,分散了对原文的关注,客观上削弱了直接阅读原文的能力,平庸的译文还让人很快失掉兴趣。与其把精力耗费在推敲译文上,不如把注释做得扎实一些,读者自然就能贯通文意。

《一瓶论语》的译,主要是对字词、结构的解释,也就是注。偶尔也有成句的翻译,有时是为了补足句子成分,有时是疏通容易误解的结构,有时只想表达得灵活一些,建议你自己多试试,看会不会译得更好。

《一瓶论语》的评,主要是对一句话或者一整章内容的感受:它在孔子当时是不是合理?放在现代有什么意义?这个部分直接反映了我的偏好与局限,所以比较谨慎,以免喧宾夺主,干扰你的独立思考。


\lypdfbookmark\section*{体例}
%%%%%%%%%%%%%%%%%%%%%%%%%%%%%%%%%%%%%%%%%%%%%%%%%%%%%%%%%%%%

\lylabel{lunyuversion}
本书的《论语》原文与杨伯峻先生《\lylink{lunyuyizhu}{论语译注}》的文本及分章相同,只是段落标点略有差异。每章的内容可以看成瓶子里的一“滴”,单独放在一个大圆角框里,框外面的左上角标有这一章的序号。框的顶部是原文,用\colorbox{lytextbackground}{浅灰底}高亮,随后就是译和评。不少“大滴”是跨页的,但是每章的原文都不会跨页,保证了完整的视觉感受。
% NOTE: 传统上,《论语》分为20篇,从“学而”到“尧曰”,每篇都用开头比较有意义的两三个字作为篇名。每篇包含若干章,多的有40多章,少的仅有3章,共512章。每章的内容有长有短,一般都不太长。大部分是记录孔子的言行,所以称为语录体。篇的排列顺序,传统上标为“学而第一”、“为政第二”等等,排列起来很不工整,只是看起来古色古香而已,而且影响PDF书签的效果,所以本书不取,改为“第一篇 学而”等等,PDF书签则用阿拉伯数字。TODO: move to 源流。

每条译用一个圆点•开头,需要解释的原文以\lyterm{黑体}显示,注解中的关键词以\lykw{楷体}显示。注解一般按原文中出现的顺序排列,有时为了处理方便,也会稍微调整前后位置,或者把有关联的几处合并为一条。注解一般宁详勿略,常会追溯到字的本义。有时连续用几个词作解释,通常是从自身的含义延伸到上下文中的含义,可以留意它们的细微差别。用“指”标出的,总是上下文中的特定含义。在我感觉“刚好合适”的地方,还会用简单的英语作注,也许能更好地体会构词和含义上的道理。如果遇到难字没有解释,多半是以前已经解释过了,可以从头搜索一下。
% NOTE: 本义主要参照《汉字源流字典》的解释。特别古老很不直观的本义,酌情采用《辞源》的第一义。

比较浅显易懂的章句不带评。有的评直接连在注解之后,有的放在整章末尾,前面不加圆点。偶尔有个人想法较多的评,则用横线隔开。总之是以读着通顺、用着方便为准。
% NOTE: 某些地方包含着我对传统文化的感情,当然,假如还能看出点别的有用的东西,更是意外之喜。

既然去掉了逐句翻译,省下来的篇幅就被用来进一步解释原文,我称之为“例参体”。我上学的时候,不太敢直接相信字典里的解释,要找到一个用法相同的典型例子才放心。我在很多注解后面,就附上了这样的例证,用【例】醒目地标出。为了更好地理解原文的某个观点或者一整章的意义,还用【参】补充了一些值得参考的观点,或者有帮助的背景知识。字词用法的例证尽量引用《论语》后文,作为预习;观点上的参考尽量引用《论语》前文,作为复习,除非遇到关键的概念,或者需要对照辨析的情况。例、参引用的原文,用\lyq{仿宋体}显示,前后不加引号。引文中如果遇到难字,就在随后的(~)里加上简注。引文中的省略号,除非另有说明,都是对原文的省略。引文\lyq{〔~〕}里的话,是为了补全句子或者插入更正。

例、参引用的内容,大部分是文言文。除了《论语》不同章句的相互参照以外,我尽量选取有代表性或者有意思的片段。它们有的支持了《论语》的观点,有的略有不同,有的则针锋相对。对于喜欢独立思考的读者,这种参差碰撞似乎比“纯净化”更有营养。但这种方式受限于我的阅读范围,在本书维护过程中,我会随时把更恰当的内容增补进来。古义今用是我比较留意的情况,所以也会用现代汉语的成语和常用词作为例证。另外还引用了少量英文材料,大都提供了汉语译文。其中,《圣经》的英文采用英王钦定本(\emph{King James Bible}),中文采用和合本,其它除了标明译者的几处,都是我翻译的。

可是过犹不及,引用与背景太多,会有明显的不平衡感,也影响对原文的专注。\textbf{建议始终把理解原文放在首位},额外的内容不妨跳读。我也会把某些背景材料逐渐转移到正在撰写的导言中,以后Web版的形式就会灵活得多。

另外,注解中带有一些语法方面的内容。中国古代强调熟读成诵、文义自达,不太看重形式化规则,汉语语法是清末开始受西方影响才逐渐建立的。语法很实用,但汉语(包括古汉语)有它自身的特点,比如往往一个字就是一个词,词性又灵活多变,名词、动词、形容词、副词等等似乎无所不能,不像英语那样有明确的对错之分。所以,既要善用语法的分析能力帮助理清头绪,也应该清楚,目前的语法体系还未能完善地概括汉语的特征。

可能你已经注意到,这本书使用了很多超链接。这是电子版的一大优势,可以突出重点,节省篇幅,更好地形成概念网络。比如,有时单解释一个词感觉不够充分,还会链接到它更基本的含义。要有效使用本书,先得熟悉跳来跳去的操作,在跳之前要留意当前的位置,阅读器也应该有后退、前进的功能。使用的超链接主要有3种样式:圆角小框跳转到某章的开头,例如 \lyref{1.1};\lyuline{实下划线}跳转到本书中的某个位置;\lydashuline{虚下划线}使用浏览器打开网址链接。\lyconvskip{此外,封面上的思维导图和联系方式也加了超链接,只是用\textcolor{\lycoverlinkcolor}{不同的颜色}表示。}直接写出的网址就不加任何效果。

自古以来,《论语》的注解可谓千锤百炼,但我有时仍对旧注无法满意,思来想去,还是按自己的理解写了出来,记为“与各家不同”。在这方面,我很不想标新立异,只是觉得自己的想法确实好一些,恳请方家不吝指正。显然,这些解释未必是我第一个想到了写出来的,只是没有精力穷举相关的材料。它们有的与旧注差别很大,有的只是细节上的不同,共计8处:\lyref{1.13} \lyref{3.4} \lyref{3.23} \lyref{8.3} \lyref{8.9} \lyref{12.1} \lyref{17.1} \lyref{17.16}。

其它的部分,还包括\lylink{characters}{人物表}、\lylink{topics}{主题索引}以及\lylink{references}{参考材料},它们的开头各有说明。这里值得一提的是,人物表把《论语》中的孔子弟子与常见人物(包括各种称呼)集中起来撰写小传,辅以超链接,就很容易熟悉各人的面貌。主题索引是从基本概念的角度梳理的《论语》纲要。另外还打算添加一个导言,用另一种方式把离散的水滴串成有机的整体,这是后续的一个重点内容。

本书不带脚注。脚注总让我把视线从正文移开,指望看到些好东西,往往却很失望,还得努力接上中断的思维。脚注只有恼人的数字,而超链接有文本作指示,是可以预期的。我的选择是,把通常作为脚注的内容,写成\lylink{sourcecode}{源代码}的注释,用“NOTE”标出。它们是对正文的补充说明,但还不值得出现在书面上。如果对正文的说法有疑惑,可以先看看注释里有没有解释。用“TODO”标出的,是我随时记下的想法,大部分很零乱,但也是后续撰写的大致方向。

% NOTE: 还有很多细节考虑,比如注释版、白文版的页眉都是插入图像曲线而不是文本字符,搜索的时候就不会定位到页眉上,选择复制也会工整很多。


\lypdfbookmark\section*{电子版}
%%%%%%%%%%%%%%%%%%%%%%%%%%%%%%%%%%%%%%%%%%%%%%%%%%%%%%%%%%%%

PDF电子版的设计,重点考虑了主流智能手机横屏阅读的方便。如果在电脑上按屏幕宽度显示,文字可能比较大。大字容易缩小,要想在手机上放大了挪来挪去,就很费劲了。内文尽量采用大小相同的字号,也是为了在小屏幕上看起来省力。内文的目录只列了顶层大纲,PDF的书签视图则详细得多,可以直接跳转到每一章。

前面说过,阅读器的选择很重要。在手机上,它最好还能自动或手动地切除白边,更有效地利用屏幕宽度。Android手机上,我喜欢用免费的 \lyurl{https://play.google.com/store/apps/details?id=org.ebookdroid}{EbookDroid},只是它的设置略微复杂。iPhone上,我印象中 \lyurl{http://www.goodreader.com}{GoodReader} 的效果很好,也不算贵。电脑上的选择就很多,显示最棒的当然是 \lyurl{https://get.adobe.com/reader/}{Adobe Reader}(免费软件)和 \lyurl{https://acrobat.adobe.com/us/en/products/acrobat-pro.html}{Adobe Acrobat}(商业软件)。

\lylabel{sourcecode}本书采用LaTeX排版,借助Python进行简单的自动化处理。主要字体有:英文和数字用 \lyurl{http://www.linuxlibertine.org/index.php}{Linux Libertine O}(公有领域),拼音用 \lyurl{https://github.com/adobe-fonts/source-han-sans}{\lypy{Source Han Sans SC}}(开源),普通正文为方正书宋,黑体字为\lyterm{方正黑体},楷书为\lykw{方正楷体},仿宋体为\lyq{方正仿宋}。后4款字体均可以\lyurl{http://www.foundertype.com/index/release\_info.html}{免费使用}。

\lylabel{projectinfo}
本书基于“知识共享”\lylink{abolylicense}{协议},以开放源代码的形式发布在GitHub上:\lyurladdr{https://github.com/abolybook/aboly}{github.com/abolybook/aboly},欢迎随时提供问题与建议。\lyconvskip{从这个PDF文件的文档属性里,可以看到它的完整版本信息:\\
\versioninfoaboly\\
其中,\projectversionnumber 是它的版本号;build.\buildnoaboly 表示从项目开始算起,它在我这里总共生成了\buildnoaboly 遍;commit.\lycommitno 是当前版本的源代码在项目历史中的编号。}

其它联系方式:
\begin{lyitemize}
\item 本书网站:\lyurladdr{http://www.abolybook.org/}{www.abolybook.org}。我暂时还没有时间把它弄得像模像样,但也已经积累了一些有意思的想法。请让我把它慢慢做好。
\item 电子邮件:\lyurladdr{mailto:abolybook@gmail.com}{abolybook@gmail.com}。
\item 新浪博客:\lyurladdr{http://blog.sina.com.cn/abolybook}{blog.sina.com.cn/abolybook}。
\item 微信订阅号:abolybook。
\end{lyitemize}


\lypdfbookmark\section*{计划中}
%%%%%%%%%%%%%%%%%%%%%%%%%%%%%%%%%%%%%%%%%%%%%%%%%%%%%%%%%%%%

本书是用业余时间断续撰写的,有很多明显可以改进的地方。下面的已知缺点,我会逐一改进,未知的问题,就需要大家的智慧与帮助。谢谢!
\begin{lyenumerate}
\item 各章之间的联系比较散乱,需要重新梳理。
\item 主题索引也应该更有条理。
\item 补充参考材料中没来及写的点评。
\item 生硬的语句会重新组织,错误会不断修正。这4条,都是0.3版的任务。
\item “评”的部分目前明显偏弱,需要补强。这是0.4版的任务。
\item 正文前面缺少一篇提纲挈领的导言,很快就会开始撰写。我对它的期望很高,考虑的内容也不少,希望不会篇幅太长。相对容易写的前面2/3,是0.5版的任务。比较难写的后面1/3,是0.6版的任务。
\item 适当配上直观的插图,代码中已经标出了若干处。我很羡慕画画好的人,封面的设计很大程度上是藏拙的办法。这个只能慢慢来。
\item 多看点书,包括国外著作,细水长流地提高总体质量。但要控制篇幅。
\item 网站。这也是长期任务,会有它单独的计划。
\end{lyenumerate}

\lycenterpar{敬请期待!}
