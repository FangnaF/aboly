\lyalt{\lylabel{characters}}{}
\phantomsection
\chapter*{人\quad 物\quad 表}
\addcontentsline{toc}{chapter}{人物表}
%%%%%%%%%%%%%%%%%%%%%%%%%%%%%%%%%%%%%%%%%%%%%%%%%%%%%%%%%%%%

% TODO: 参照《孔子家语·七十二弟子解》。
% https://zh.wikipedia.org/wiki/Category:%E5%AD%94%E5%AD%90%E5%BC%9F%E5%AD%90

% NOTE: Slingerland书附录2、3的结构与这里很像,是让我很开心的巧合。非孔子弟子的人物如果仅见于某一章,我就直接在那一章介绍,这样处理似乎更顺一些。

\lypdfbookmark\section*{孔子弟子}
%===========================================================

孔子究竟有多少弟子?由于年代久远,还没有一个确切的说法。\lylabel{qishizi}《史记·孔子世家》载:\lyq{孔子以诗书礼乐教,弟子盖三千焉,身通\lylink{liujing}{六艺}者七十有二人。}似乎有所夸大。《史记·仲尼弟子列传》则称\lyq{孔子曰:“受业身通者七十有七人”},并根据当时(约前90年)可考的文献,依次列出了姓名,或详或简地记述了其中29人的事迹。

以下收录了《论语》中出现过,并大致公认属实的孔子弟子,也是29人,按拼音排序。比《仲尼弟子列传》增加的是\lycharlink{lao}{牢}和\lycharlink{shencheng}{申枨},缺少的是公晳哀(齐国人,字季次,因\lyq{天下无行}而\lyq{未尝\lylink{shi4a}{仕}})和商翟(字子木,鲁国人,少孔子29岁,是孔子《\lylink{yijing}{易}》学的传人)。

本书统一采用古代习俗计算年龄,只要在某年生活过,就算作一岁。孔子生于前551年(\lycharlink{xianggong}{鲁襄公}\,22年),卒于前479年(\lycharlink{aigong}{鲁哀公}\,16年),寿73岁。遵照传统,将诸弟子的出生年份统一记为“少(\lypy{shào})孔子多少岁”。
% NOTE: 年龄等于当年年份减去出生年份后加一。B-A+1的算法,是按古代纪年算的。旧历年与公历年不很吻合,如果用公历年计算,可能有1岁左右的误差。“习俗”的说法见《吴宓日记续编》第1册45页。

\bigskip

\lylabel{boniu}
\lycharname{伯牛},\lycharname{冉伯牛}:冉耕,字伯牛,鲁国人,少孔子7岁,\lylink{11.3}{孔门十哲}之一,患病早逝。% TODO: 定陶三冉?

\lylabel{fanxu}
\lycharname{樊须},\lycharname{樊迟}:樊须,字子迟,鲁国人,少孔子46岁。在前484年春的\lylink{jiquzhizhan}{稷曲之战}中,\lycharlink{ranqiu}{冉求}统领鲁国左路军,樊迟担任他的车右武士,建议冉求反复申明号令再冲锋,果然士气大振,赢得了胜利。他是孔子晚年返鲁后收取的学生。
% NOTE: 从《孔子家语》及杨伯峻说,非少36岁,史记误;见《左传·哀公十一年》:\lyq{须也弱}。

\lylabel{gongboliao}
\lycharname{公伯寮}(\lypy{liáo}):字子周,鲁国人。也有根据 \lyref{14.36} 认为他并非孔子弟子。

\lylabel{gongxihua}
\lycharname{公西华},\lycharname{子华},\lycharname{赤}:公西赤,字子华,鲁国人,少孔子42岁。他娴于交际,精通礼仪,据《礼记·檀弓上》,孔子的丧礼就是由他设计装饰的。《淮南子·齐俗训》比较了他和\lycharlink{zengshen}{曾参}奉养父母的特点:\lyq{公西华之养亲也,若与朋友处;曾参之养亲也,若事严主烈君——其于养一也。}
% NOTE: 《檀弓上》:\lyq{孔子之丧,公西赤为志焉:……}“志”,钱玄解作主持操办,杨天宇解作墓志铭,似均不确。主持者应为子贡。% TODO: “设计装饰”搭配不顺。

\lylabel{gongyechang}
\lycharname{公冶长}(\lypy{cháng}):字子长,鲁国人。传说他懂鸟语,\lyq{曾读百鸟之书,能闻众鸟之音}(《公冶长听鸟语纲常》),因此蒙冤入狱,见皇侃《论语义疏》引文:\lyq{主问冶长何以杀人,冶长曰:“解鸟语,不杀人。”主曰:“当视之。若必解鸟语,便相放也,若不解,当令偿死。”驻冶长在狱六十日……}这虽然离奇,但历来广为人知,成为典故使用,如白居易《池鹤八绝句》诗序:\lyq{诸禽似有所诮,鹤亦时复一鸣。予非冶长,不通其意。}

\lylabel{lao}
\lycharname{牢}:《论语》中身份最隐秘的弟子,据《孔子家语》王肃序:孔子\lyq{弟子有琴张,一名牢,字子开,亦字张,卫人也},后世多认为是王肃的伪撰。

\lylabel{minziqian}
\lycharname{闵子骞}(\lypy{qiān}),\lycharname{闵子}:闵损,字子骞,鲁国人,少孔子15岁,\lylink{11.3}{孔门十哲}之一。他是元代编录的童蒙读物《二十四孝》第4个故事“\lylink{danyishunmu}{单衣顺母}”的主角。

\lylabel{nangongkuo}
\lycharname{南宫适}(\lypy{kuò}),\lycharname{南容}:字子容,鲁国人,孔子的\lylink{5.2}{侄女婿}。《孔子家语·弟子行》中,\lycharlink{zigong}{子贡}形容他\lyq{独居思仁,公言言义。}

\lylabel{qidiaokai}
\lycharname{漆雕开}:字子若,蔡国人,少孔子11岁,是孔子之后的\lylink{zizhang}{八儒}之一。《孔子家语·七十二弟子解》说他\lyq{习《\lylink{shangshu}{尚书}》,不乐仕。}《韩非子·显学》说他方正刚直,\lyq{不色挠,不目逃,行曲则违于臧获,行直则怒于诸侯,世主以为\lylink{lian2}{廉}而礼之。}(挠:屈。目逃:避视。违:避。臧获:奴婢的贱称,骂奴为臧,骂婢为获。)

\lylabel{ranqiu}
\lycharname{冉求},\lycharname{冉有},\lycharname{冉子},\lycharname{求}:冉求,字子有,鲁国人,少孔子29岁,\lylink{11.3}{孔门十哲}之一。他多才多艺,精明务实,曾跟随孔子周游,后来成为\lycharlink{jikangzi}{季康子}的得力家臣,不久孔子就被迎接返鲁了。直到前472年(《左传·哀公二十三年》),还有他奉季康子之命赴宋国吊唁的记录。

\lylabel{shencheng}
\lycharname{申枨}(\lypy{chéng}):字周,鲁国人,事迹不详。

\lylabel{simaniu}
\lycharname{司马牛}:司马耕,向氏,字子牛,宋国人,是\lycharlink{huantui}{桓魋}的弟弟,因为桓魋任宋国司马,故又以司马为氏。《仲尼弟子列传》说他\lyq{多言而躁},可能是根据 \lyref{12.3} 的推想。桓魋叛乱失败后,司马牛退还宋国的封地,辗转于齐、吴、宋,最终死在鲁国城外。

\lylabel{tantaimieming}
\lycharname{澹(\lypy{tán})台灭明}:字子羽,鲁国\lylink{wucheng}{武城}人,少孔子39岁。《仲尼弟子列传》说他相貌丑陋,曾被孔子认为资质低劣。他退而修行,南游至楚,有弟子三百,讲究\lylink{4.5}{取予}\lylink{qujiu}{去就},名动诸侯。\lyq{孔子闻之曰:“吾以言取人,失之\lycharlink{zaiyu}{宰予};以貌取人,失之子羽。”}《孔子家语·弟子行》中,\lycharlink{zigong}{子贡}形容他\lyq{贵之不喜,贱之不怒,苟利于民矣,廉于行己。}今江苏省苏州市吴中区的澹台湖,据说就是他曾居住之地。
% NOTE: 《韩非子·显学》对子羽之貌、宰予之智有相反的说法:\lyq{澹台子羽,君子之容也,仲尼几而取之,与处久而行不称其貌;宰予之辞,雅而文也,仲尼几而取之,与处久而智不充其辩。故孔子曰:“以容取人乎,失之子羽;以言取人乎,失之宰予。”}相当于绣花枕头中看不中用的意思。

\lylabel{wumaqi}
\lycharname{巫马期}:巫马施,字子期,陈国人,少孔子30岁。《吕氏春秋·开春论·察贤》说他也曾任\lylink{shanfu}{单父}宰,并比较了他与\lycharlink{zijian}{子贱}之政的区别:\lyq{宓子贱治单父,弹鸣琴,身不下堂,而单父治。巫马期以星出,以星入,日夜不居,以身亲之,而单父亦治。巫马期问其故于宓子,宓子曰:“我之谓任人,子之谓任力。任力者故劳,任人者故逸。”宓子则君子矣。……巫马期则不然,弊生事精,劳手足,烦教诏,虽治,犹未\lylink{zhi4e}{至}也。}(事:使,用。)

\lylabel{yanhui}
\lycharname{颜回},\lycharname{颜渊},\lycharname{回}:颜回(前521年--前481年),字子渊,鲁国人,少孔子30岁,\lylink{11.3}{孔门十哲}之一。他尊师好学,乐仁守礼,最得孔子推重期许,也受到同门的一致敬佩。可惜天不假年,29岁就熬白了头,41岁赍志而殁(早于孔子2年),未能传承光大师学,令孔子伤痛欲绝。《仲尼弟子列传》载,\lyq{孔子\lylink{11.10}{哭之恸},曰:“自吾有回,门人益亲。”}后世尊其为孔门众贤之首,封为\lykw{复圣}。颜回在古籍中常以孔子代言人的形象出现。唐代名臣、书法家颜真卿是他的40世孙。% NOTE: 不是31岁去世。明朝编纂的颜氏家族志《\lylink{6.11}{陋巷}志》说颜回\lyq{娶宋戴氏,生子歆}。

\lylabel{yanlu}
\lycharname{颜路}:颜无繇(\lypy{yóu}),字路,鲁国人,\lycharlink{yanhui}{颜回}之父,少孔子6岁,是孔子的早期弟子。

\lylabel{yong}
\lycharname{雍},\lycharname{仲弓}:冉雍,字仲弓,鲁国人,少孔子29岁,\lylink{11.3}{孔门十哲}之一。《孔子家语·弟子行》中,\lycharlink{zigong}{子贡}形容他\lyq{在贫如客,使其臣如借,不\lylink{qiannu}{迁怒},不深怨,不录旧罪。}(如客:形容庄重自持,不因贫降志。如借:形容谨慎有礼,不自居其位。录:记住。)如果《荀子》提到的“子弓”就是冉雍(例同仲由字子路,又称季路),那么他在当时很受尊崇,与孔子并列,被称为\lyq{大儒}、\lyq{圣人之不得势者}。% NOTE: 《荀子》之说,见《非相》、《非十二子》、《儒效》篇。

\lylabel{youruo}
\lycharname{有若},\lycharname{有子}:字子有,鲁国人,少孔子33岁,《孔子家语·七十二弟子解》称他强识好古。《孟子·滕文公上》、《仲尼弟子列传》载,孔子去世后,众弟子思慕不已,因为有若的气质见识与孔子相似,\lycharlink{zixia}{子夏}、\lycharlink{ziyou}{子游}、\lycharlink{zizhang}{子张}等就奉他为师,但有若之才德不足以服众,这种偶像崇拜很快被放弃了。《礼记·檀弓下》载,\lycharlink{aigong}{鲁哀公}曾向他问礼,他去世后,继任的鲁悼公(前467年--前437年在位)亲自前来吊唁。他在《论语》中出现的4次,都是传道授业的口吻,除 \lyref{12.9} 外,称谓都是有子,待遇仅次于\lycharlink{zengshen}{曾参}。

\lylabel{yuansi}
\lycharname{原思},\lycharname{宪}:原宪,字子思,少孔子36岁。《仲尼弟子列传》载,孔子去世后,他在乡间隐居,生活极清贫。\lycharlink{zigong}{子贡}乘驷马高车来探望他,\lyq{宪\lylink{she4}{摄}敝衣冠见子贡。子贡耻之,曰:“夫子岂\lylink{bing4}{病}乎?”原宪曰:“吾闻之:无财者谓之贫,学道而不能行者谓之病。若宪,贫也,非病也。”子贡惭,不怿而去,终身耻其言之过也。}(怿(\lypy{yì}):喜悦,高兴。)《史记·游侠列传》说他\lyq{读书怀独行君子之德,义不\lylink{gouhe}{苟合}当世,当世亦笑之。……终身空室蓬户,褐衣\lylink{shushi}{疏食}不厌。死而已四百余年,而弟子志之不倦。}(褐衣:粗布衣。) % NOTE: 《列传》说原思隐居于卫,《韩诗列传》、《庄子》说于鲁。

\lylabel{zaiyu}
\lycharname{宰予},\lycharname{宰我},\lycharname{予}:字子我,鲁国人,少孔子29岁,\lylink{11.3}{孔门十哲}之一。他口才出众,甚至超过大名鼎鼎的外交家\lycharlink{zigong}{子贡};思维活跃,不仅\lylink{17.21}{质疑}过三年丧期的必要,还怀疑过黄帝的传说,向孔子追问\lycharlink{wudi}{五帝}的由来(《大戴礼记·五帝德》)。可孔子是反感伶牙俐齿的,也不太欣赏出格的想法和行为,《论语》中宰予总是受到敲打。《孟子·公孙丑上》引用他对孔子的评价:\lyq{以予观于夫子,贤于\lycharlink{yao}{尧}、\lycharlink{shun}{舜}远矣},认为他和\lycharlink{zigong}{子贡}、\lycharlink{youruo}{有若}一样\lyq{智足以知圣人}。

\lylabel{zengxi}
\lycharname{曾皙}(\lypy{xī}),\lycharname{点}:曾点,字皙,\lycharlink{zengshen}{曾参}之父,鲁国\lylink{wucheng}{南武城}人。《孟子·尽心下》说,他属于孔子眼中的\lylink{kuangjuan}{狂士}。他教子严厉,有一次曾参在瓜田里除草,不小心斩到瓜根,他勃然大怒,操起大棒抡在儿子背上。曾参倒地昏厥良久,醒来后连忙好声好气跟父亲陪不是,还问有没有用力过猛伤到身体。事后,孔子批评曾参之孝\lylink{11.16}{过犹不及}:\lyq{(\lycharlink{shun}{舜}事父)小棰则待过,大杖则逃走。……今参事父,委身以待暴怒,殪而不避,既身死而陷父于不义,其不孝孰大焉?}(棰(\lypy{chuí}):棍棒,用作动词。待过:等着接受责罚。殪(\lypy{yì}):死。)事见《孔子家语·六本》、《吕氏春秋·孟夏纪·劝学》。后人不辨因果,附会出“棍棒底下出孝子”的俗语。

\lylabel{zengshen}
\lycharname{曾子},\lycharname{参}(传统读\lypy{shēn},也有根据其\lylink{chenghu}{字},认为应读\lypy{cān},即同“骖”):曾参(前505年--前435年),字子\lylink{yu2b}{舆},鲁国\lylink{wucheng}{南武城}人,少孔子46岁。他是孔子之学的主要\lycharlink{zisi}{传承者},后被封为\lykw{宗圣}。曾参事父母至孝,是《二十四孝》第3个故事“啮指心痛”的主角。\lylabel{xiaojing}儒家十三经之一的《孝经》,就采用他与孔子问答孝道的形式,共18章,约1800字,各章多引《诗经》作为总结。曾参笃于礼义,《礼记·檀弓上》载,他病重卧床的那晚,听侍童说自己睡的是\lylink{qingdafushi}{大夫}之席(曾参为\lylink{qingdafushi}{士}),便挣扎着起身要求换席。两个儿子劝他等到天亮再换,他对儿子说:\lyq{尔之爱我也不如彼。君子之爱人也以德,细人之爱人也以姑息。吾何求哉?吾得正而毙焉,斯已矣。}(彼:指侍童。细人:见识短浅的人。姑息:苟安。)换席后还没躺稳当,他就去世了。《大戴礼记》卷四、五共10篇,是他的语录,朱熹称他是《大学》的作者,但并没有给出根据。
% NOTE: 朱熹《四书章句》对《礼记·大学》作了改编与划分,“未之有也”及之前的205字称为“经”,之后的1546字称为“传”,认为是解释“经”的,并加说明:\lyq{右经一章,盖孔子之言,而曾子述之;其传十章,则曾子之意而门人记之也。旧本颇有错简,今因程子所定,而更考经文,别为序次如左。}

除了孔子的话以外,《论语》中对他所有的称谓都是曾子。其他弟子虽也有以\lylink{zi3}{子}相称的,包括\lycharlink{youruo}{有若}、\lycharlink{ranqiu}{冉求}和\lycharlink{minziqian}{闵子骞},但都没有这样的待遇。后世多认为曾参和有若的弟子是编订《论语》的主力。% TODO: 1.2有子,1.4曾子,补充入源流。

\lylabel{zigao}
\lycharname{子羔},\lycharname{柴}:高柴,字子羔,少孔子30岁,\lylink{kangshu}{卫国}人,曾任卫国\lylink{shishi}{士师}。《孔子家语·七十二弟子解》说他\lyq{长不过\lylink{chi3}{六尺},状貌甚恶;为人笃孝而有法正。}《礼记·檀弓上》记载他\lyq{执亲之丧也,泣血三年,未尝见齿,君子以为难。}(见齿:笑。)《孔子家语·致思》载,他在\lylink{kuaikuizhiluan}{蒯聩之乱}出逃时,有位曾被他处以断足之刑的守门人连续3次为他指路,并坦言自己受刑罪有应得,对子羔是心悦诚服。孔子得知后赞许道:\lyq{善哉为吏!其用法一也。思仁恕则树德,加严暴则树怨。公以行之,其子羔乎!}

\lylabel{zigong}
\lycharname{子贡},\lycharname{赐}:端木赐,字子贡,少孔子31岁,卫国人,\lylink{11.3}{孔门十哲}之一。《仲尼弟子列传》说,\lyq{子贡利口巧辞,孔子常黜其辩。}(黜:贬抑。)他有杰出的外交才干,\lyq{子贡一出,存鲁、乱齐、破吴、强晋而霸越。子贡一使,使势相破,十年之中,五国各有变。}他有卓越的经商本领,善于观察行情低买高卖,家累千金,是同门中的首富。他是孔子最亲密的弟子之一,老师病重时他赶来探望,孔子问:“\lyq{赐,汝来何其晚也?}”孔子去世后,他是丧礼的主持人,众弟子服丧3年,唯有他筑庐冢边,守丧6年。\lyref{19.21} 至 \lyref{19.25} 反映了子贡对老师的拳拳仰慕之心。

司马迁特别偏爱子贡,《仲尼弟子列传》用了近1/3篇幅描写他游说齐、吴、越、晋之间,使鲁国免遭战乱的功绩。《货殖列传》评价:\lyq{夫使孔子名布扬于天下者,子贡先后之也。此所谓得势而益彰者乎?}

\lylabel{zijian}
\lycharname{子贱}:宓(\lypy{mì},旧读\lypy{fú})不齐,字子贱,少孔子30岁,鲁国人,曾任\lylabel{shanfu}鲁国单父(\lypy{shàn fǔ},也作亶父,今山东省菏泽市单县)邑宰。据《吕氏春秋·审应览·具备》,上任之初,他通过隐喻巧妙地获得了鲁君的信任,使单父大治,流传下“掣肘”的典故。《说苑·政理》载,孔子问他何以\lyq{治单父而众说},子贱说自己爱民:\lyq{父其父,子其子,\lylink{xu4c}{恤}诸孤而哀丧纪},孔子认为是小节:\lyq{小民附矣,犹未足也};子贱说自己待人谦恭推诚,孔子认为是中节;子贱说访求到5位能于自己的贤人协理政务,孔子赞叹:\lyq{欲其大者,乃于此在矣。昔者\lycharlink{yao}{尧}、\lycharlink{shun}{舜}清微其身,以听观天下,务来贤人。夫举贤者,百福之宗也,而神明之主也。不齐之所治者小也,不齐所治者大,其与尧、舜继矣。}《史记·滑稽列传》末尾评论:\lyq{\lycharlink{zichan}{子产}治郑,民不能欺;子贱治单父,民不忍欺;西门豹治邺,民不敢欺。三子之才能谁最贤哉?辨治者当能别之。}(\lylabel{ximenbao}西门豹:战国初期魏国邺县(在今河北省邯郸市临漳县西南)县令,破除陋习,引水灌田,\lyq{名闻天下,泽流后世}。)据《颜氏家训·书证》,今文《\lylink{shangshu}{尚书}》的传人秦博士伏生,就是子贱的后人。% TODO: 解释 博士 in 源流。

\lylabel{zilu}
\lycharname{子路},\lycharname{季路},\lycharname{仲由},\lycharname{由}:仲由(前542年--前480年),字子路,又称季路,鲁国\lylink{bianyi}{卞邑}人,\lylink{11.3}{孔门十哲}之一,少孔子9岁。除了\lycharlink{yanlu}{颜路}和\lycharlink{boniu}{伯牛}以外,他是《论语》诸弟子中最年长的。子路忠勇直率,长期追随孔子左右,虽常因为思维简单挨孔子的批评,但也只有这个大龄\lylink{13.3}{野小子}最能让老师放松随意吧!子路少时家贫,成年后热心功名,曾任\lycharlink{jihuanzi}{季桓子}的家宰,后来又任卫大夫\lycharlink{kongkui}{孔悝}的\lylabel{puyi}蒲邑(在今河南省长垣县)邑宰。《孔子家语·辩政》说,子路治蒲三年,孔子路过观察民风,三称其善。\lylabel{weiqinfumi}身处富贵之时,他想起故去的父母,\lyq{乃叹曰:“虽欲食藜藿之食,为亲负米,不可得也!”}(藜藿(\lypy{lí huò}):粗劣的饭菜。见《二十四孝》第5个故事“为亲负米”。)前480年(《左传·哀公十五年》),63岁的子路奋勇\lylink{kuaikuizhiluan}{营救}被\lycharlink{weijun}{蒯聩}挟持的主公孔悝,孤身不敌,帽缨(系在下巴上固定帽子的束带)也被击断了。在生命最后一刻,\lyq{子路曰:“君子死,冠不免。”结缨而死。}死且不忘礼,不愧为孔子的学生。
% NOTE: “为亲负米”事见《说苑·建本》:\lyq{昔者由事二亲之时,常食藜藿之实,而为亲负米百里之外。亲没之后,南游于楚,从车百乘,积粟万钟,累茵而坐,列鼎而食。愿食藜藿、为亲负米之时,不可复得也!}

《礼记·檀弓上》载,孔子正在院子中间恸哭子路,听说其尸体被施以醢(\lypy{hǎi},剁成肉酱)刑的惨状,立刻让把家里的肉酱倒掉。4年之间,\lycharlink{li}{孔鲤}、\lycharlink{yanhui}{颜回}、子路接连亡故,对孔子的打击可想而知,一年后他也去世了。

\lylabel{zixia}
\lycharname{子夏},\lycharname{商}:卜商,字子夏,晋国温邑(今河南省焦作市温县)人,少孔子44岁,\lylink{11.3}{孔门十哲}之一。他家境清贫,笃学精思,敬谨尚贤,《史记·孔子世家》在记述孔子笔削《春秋》时特别说,\lyq{子夏之徒不能赞一辞}(赞:助),可见他在诸弟子中的理论水平很突出。《后汉书·徐防传》引徐防上疏:\lyq{臣闻《诗》《书》《礼》《乐》,定自孔子;发明章句,始于子夏。}(发明:阐述意旨。章句:分章断句解读。)孔子去世后,他到魏国西河(在今山西、陕西黄河交界地带)讲学,魏文侯(前445年--前396年在位)师事之,咨问国政,《礼记·乐记》记有他们关于古乐和新乐的问答。他所创立的西河学派,常被认为是由儒入法的重要环节。子夏晚年因哭子之丧而失明,事见《礼记·檀弓上》。据说今本《\lylink{shijing}{诗经}》由他而传,《春秋公羊传》、《春秋谷梁传》的作者也都是他的学生。
% NOTE: 由儒入法的重要环节:子夏的弟子李克,有认为就是法家始祖李悝;据信荀子也与子夏之儒颇有渊源。

\lylabel{ziyou}
\lycharname{子游},\lycharname{言游},\lycharname{偃}:言偃,字子游,\lylink{taibo}{吴国}人,少孔子45岁,\lylink{11.3}{孔门十哲}之一。他是孔子知名弟子中唯一的南方人,今江苏省常熟市留有言子墓等遗迹。他的女婿是\lycharlink{zizhang}{子张}的儿子申祥。

\lylabel{lijiliyun}
《礼记·礼运》通过子游与孔子问答的形式,描绘了儒家理想中的\lykw{大同}社会:\lyq{大道之行也,天下为公,选贤与能,讲信修睦。故人不独亲其亲,不独子其子;使老有所终,壮有所用,幼有所长,矜寡、孤独、废疾者皆有所养;男有分,女有归。货,恶其弃于地也,不必藏于己;力,恶其不出于身也,不必为己。}(\lylabel{guan1a}矜:通“鳏”(\lypy{guān}),成年男子无妻或丧妻。)接着指出,在\lyq{天下为家}的时代,如能\lyq{礼义以为纪,以正君臣,以笃父子,以睦兄弟,以和夫妇},仍不失为\lykw{小康}。然后重点论述了礼义对于维系社会的价值:\lyq{故欲恶者,心之大端也。人藏其心,不可测度也。美恶皆在其心,不见其色也,欲一以穷之,舍礼何以哉?……故礼义也者,人之大端也,……故唯圣人为知礼之不可以已也。故坏国、丧家、亡人,必先去其礼。}
% TODO: 据传孔子云:“吾门有偃,吾道其南”,但不见于《孔子集语》,未知出处。

\lylabel{zizhang}
\lycharname{子张},\lycharname{师},\lycharname{张}:颛(\lypy{zhuān})孙师,字子张,鲁国人,少孔子48岁。《论语》诸弟子中数他最年轻,其次是\lycharlink{zengshen}{曾参}和\lycharlink{fanxu}{樊须}。《韩非子·显学》称,孔子之后儒分八派,“子张之儒”被排在第一位,超过“\lycharlink{zisi}{子思}之儒”,可惜今已不传。他的祖先是和\lycharlink{chenchengzi}{陈公子完}一同逃往齐国的公子颛孙(《左传·庄公二十二年》),后来又到鲁国定居,就以颛孙为氏。传到子张这一代,已经是\lyq{鲁之鄙家}了(《吕氏春秋·尊师》),还做过马贩子(《尸子·劝学》),想必有不为人知的艰辛,但他志高求进,心气很足,使年纪相仿的同门感到敬畏。据《史记·儒林列传》,孔子过世后他去了陈国居住,就是不曾忘怀遥远的血脉吧?他去世时\lycharlink{zengshen}{曾参}正在服母丧,所以是英年早逝(《礼记·檀弓下》)。病重之时,他把儿子召到身旁,像是自言自语道:\lyq{“君子\lylink{yue1a}{曰}终,小人曰死,吾今日其\lylink{shu4a}{庶几}乎?”}(《礼记·檀弓上》。)
% NOTE: 19.15、19.16古注解说不一,多有贬损子张的偏见,或由11.18臆测。既乏史实,当推以情理,“志高求进,心气很足,使年纪相仿的同门感到敬畏”,以解19.15、19.16皆通。


\lypdfbookmark\section*{其他}
%===========================================================

以下收录了在不同章中重复出现过,而又不是孔子弟子的人物,也包括季氏、孟氏这样的家族,共36条。

\bigskip

\lylabel{aigong}
\lycharname{哀公}:即鲁哀公(?--前468年),名蒋,鲁国第26任国君(“遭难已甚曰哀”),\lycharlink{dinggong}{鲁定公}之子。前494年--前468年在位,期间鲁国实权一直被\lycharlink{sanhuan}{三桓}把持,君臣离心。在位的最后一年,哀公想借助越国的力量除掉三桓,可能是计划泄露,遭到三桓联合攻击,辗转流亡于卫、邹、越,同年又被迎接(挟持?)回国,很快就去世了。事见《史记·鲁周公世家》、《左传·哀公二十七年》。《左传》的内容始于隐公元年(前722年),结于哀公二十七年(前468年)。% TODO: 末句移至“相关的几本书”。

\lylabel{boyishuqi}
\lycharname{伯夷、叔齐}:商朝末年孤竹国(在今河北省秦皇岛市卢龙县)国君的长子、三子。叔齐先被立为太子,尊长欲让与伯夷;伯夷不愿违背父意,就与叔齐一同出走,投奔\lycharlink{wen}{周文王}。当时文王已死,\lycharlink{wu}{武王}领兵伐纣,他们认为是犯上违仁之举,叩马而谏;周灭商后,他们耻食周粟,隐居于首阳山(在今甘肃省定西市渭源县),采食野菜度日,终于饿死。\lylabel{caiweige}《史记·伯夷列传》记载(拟托?)了他们临终前的悲歌:\lyq{登彼西山兮,采其薇矣。以暴\lylink{yi4e}{易}暴兮,不知其非矣。\lycharlink{sanhuang}{神农}、\lylink{tangyu}{虞}\lylink{xiachao}{夏}忽焉没兮,我安适归矣?于嗟徂兮,命之衰矣!}(于嗟:叹词。徂(\lypy{cú}):去,往,指死。)
% NOTE: 《伯夷列传》为司马迁之浩叹,可读。

\lylabel{chenkang}
\lycharname{陈亢},\lycharname{陈子禽},\lycharname{子禽}:陈亢,字子禽,陈国人。一说是孔子的学生,从 \lyref{19.25} 来看不像。

\lylabel{dinggong}
\lycharname{定公}:即鲁定公,名宋,鲁国第25任国君,\lycharlink{zhaogong}{鲁昭公}的庶弟,\lycharlink{aigong}{鲁哀公}之父。他于前509年--前495年在位,期间国政被\lycharlink{jipingzi}{季平子}、\lycharlink{jihuanzi}{季桓子}、\lycharlink{yanghuo}{阳货}、\lycharlink{shusunwushu}{叔孙武叔}把持。前501年--前498年间任用孔子,国势一度振兴。但受到齐国\lylink{18.4}{糖衣炮弹}的诱惑,荒政废礼,孔子只好出走。

\lylabel{gongshuwenzi}
\lycharname{公叔文子}:卫国大夫公叔发,也是\lycharlink{weilinggong}{卫灵公}的堂兄弟。\lylink{qirenkuinvyue}{前497年}孔子来到卫国,公叔文子就在同年或略早去世。《左传·定公十四年》载,他家境富裕,曾邀请灵公赴宴,却被\lycharlink{shiyu}{史鱼}警告:\lyq{子必祸矣!子富而君贪,罪其及子乎!}然后补充道:你自己谨守臣礼,可免于难;但你儿子公叔戍富而骄,\lyq{其亡乎?富也不骄者鲜,吾\lylink{ozhiv}{唯子之见};骄而不亡者,未之有也,戍必与焉!}不出所料,前496年春,灵公就以将害\lycharlink{nanzi}{南子}为由,把公叔戍逐去鲁国了。

\lylabel{gongzijiu}
\lycharname{公子纠}(?--前685年):\lycharlink{qihuangong}{齐桓公}小白的二哥。他受\lycharlink{guanzhong}{管仲}、召忽辅佐,得鲁国援助,与小白争立为齐君。失败后,鲁国迫于齐桓公武力,杀死了公子纠。

\lylabel{guanzhong}
\lycharname{管仲}(约前720年--前645年):春秋时期法家代表人物,名夷吾,字仲,谥号是敬(“令善典法曰敬”)。他和\lylabel{baoshuya}\lycharlink{qihuangong}{齐桓公}的师傅鲍叔牙自少年时即为至交好友,史称“管鲍之交”。他受鲍叔牙极力推荐,得到\lycharlink{qihuangong}{齐桓公}信任,被尊称为“仲父”,主持齐政40余年。期间力行改革,宽政惠民,富国强兵,使齐国雄霸于诸侯。《史记·管晏列传》、《史记·齐太公世家》简要记录了其生平事迹,西汉刘向编辑的《管子》集中反映了其思想主张。% TODO: add 对应“管仲之器小”的评论及链接?

\lylabel{jikangzi}
\lycharname{季康子},\lycharname{康子}:季孙肥(?--前468年),\lycharlink{jishi}{季孙氏}第6代宗主,\lycharlink{jihuanzi}{季桓子}之子,曾任鲁国正卿,谥号是康(“安乐抚民曰康”)。前484年,由于\lycharlink{ranqiu}{冉求}、\lycharlink{fanxu}{樊须}在\lycharlink{jiquzhizhan}{稷曲之战}立下大功,季康子派人从卫国将孔子恭迎回鲁国。从\lylink{qirenkuinvyue}{父弃}到子迎,孔子已经漂泊异乡14年了,从此开始了最后5年相对安稳的教育和著述。

\lylabel{jishi}
\lycharname{季氏},\lycharname{季孙},\lycharname{季}:即季孙氏,鲁国\lycharlink{sanhuan}{三桓}家族之一,是鲁桓公嫡次子\lycharlink{jiyou}{季友}的后代,实力为三桓之首,在鲁国权势熏天。\lycharlink{jihuanzi}{季桓子}、\lycharlink{jikangzi}{季康子}任\lylink{zongzhu}{宗主}期间,孔子的不少弟子做过季氏家臣,如\lycharlink{ranqiu}{冉求}、\lycharlink{zilu}{子路}、\lycharlink{yong}{冉雍},此外也有不愿就列的\lycharlink{minziqian}{闵子骞}。后来鲁穆公(前415年--前377年在位)继位,终于从三桓手中夺回政权。季孙氏据其封邑\lylink{feiyi}{费邑}独立为费国,后世宗主仅费惠公在《孟子·万章下》有片言留存。从鲁僖公元年(前659年)季友任\lylink{qingdafushi}{正卿}算起,季氏家族在鲁国风光了近两个半世纪。

\lylabel{kongwenzi}
\lycharname{孔文子},\lycharname{仲叔圉}(\lypy{yǔ}):孔圉(?--前480年),谥号是文,\lycharlink{weilinggong}{卫灵公}时期的卫国上卿。他的妻子是卫庄公\lycharlink{weijun}{蒯聩}的姐姐,\lylabel{kongkui}儿子孔悝(\lypy{kuī})也是卫国大夫。\lycharlink{zilu}{子路}曾任孔悝的邑宰。

\lylabel{li}
\lycharname{鲤},\lycharname{伯鱼}:孔鲤(前532年--前483年),字伯鱼,孔子唯一的儿子。出生时\lycharlink{zhaogong}{鲁昭公}送来一条鲤鱼当贺礼,就以之为名。他比孔子早去世4年,终年50岁,留下幼子孔伋。

\lylabel{zisi}
孔伋(前483年--前402年),字子思(与\lycharlink{yuansi}{原宪}同字),受业于\lycharlink{zengshen}{曾参},后被封为\lykw{述圣},\lylabel{mengzi}亚圣孟轲是其再传弟子。他继承了孔子的学统,一般认为,《礼记》第30--33篇:《坊记》、《中庸》、《表记》、《缁衣》,都是他的作品。南宋咸淳三年(1267年),宋度宗诏令以\lycharlink{yanhui}{颜回}、曾参、孔伋、孟轲配享于孔庙,随孔子接受祭祀,称为\lykw{四配},位在\lylink{sikeshizhe}{十哲}之前。
% NOTE: 孟子并非直接受业于子思,《孟子荀卿列传》为是。《史记·孔子世家》、《礼记·中庸》郑玄目录、朱熹《四书章句》,皆以子思为《中庸》作者。

\lylabel{linfang}
\lycharname{林放}:鲁国人,事迹不详。% NOTE: 相传林氏为\lycharlink{biganjiziweizi}{比干}后人,今山东省新泰市放城镇为林放故里。

\lylabel{liuxiahui}
\lycharname{柳下惠}(前720年--前621年):展获,字禽,又字季,亦称展禽或柳下季,鲁国大夫,柳下是他的封邑(在今山东省济南市平阴县),惠是他的\lylink{liuxiahuiqi}{私谥}(“柔质慈民曰惠”)。他以正直著称,事迹流传较少,但在《左传》、《国语》、《孟子》中都受到\lylink{hesheng}{推崇}。他是柳姓始祖,典故“坐怀不乱”的主角。

\lylabel{menggongchuo}
\lycharname{孟公绰},\lycharname{公绰}:与孔子同时代的鲁国大夫,\lycharlink{mengshi}{孟氏}族人,为人廉静寡欲。《左传·襄公二十五年》(前517年)载,当年春,\lycharlink{cuizhu}{崔杼}率齐军侵入鲁国北疆,襄公向晋国求救。孟公绰发现齐军不像往常那样大肆劫掠,穷尽民力,认为崔杼别有所图:\lyq{崔子将有大志,不在病我,必速归,何患焉?}果然齐军虚张声势了一阵就回去了,同年5月,齐庄公被弑。

\lylabel{mengshi}
\lycharname{孟氏},\lycharname{孟}:即孟孙氏,鲁国\lycharlink{sanhuan}{三桓}之一,是鲁桓公庶长子庆父的后代,已知的最后一代宗主是第11代的\lycharlink{mengjingzi}{孟敬子}。据说孟子也是庆父的后代,但和孟孙氏不同\lylink{zongzhu}{族}。

\lylabel{mengwubo}
\lycharname{孟武伯}:名彘(\lypy{zhì}),谥号是武,\lycharlink{mengyizi}{孟懿子}长子,\lycharlink{mengshi}{孟孙氏}第10代宗主。\lycharlink{aigong}{鲁哀公}在位的最后一年(前468年),出游时遇见孟武伯,哀公\lyq{曰:“请有问于子:余及死乎?”对曰:“臣无由知之。”三问,卒辞不对。}(及死:善终,寿终。)哀公被迫要对\lycharlink{sanhuan}{三桓}动手了。事见《左传·哀公二十七年》。

\lylabel{qihuangong}
\lycharname{齐桓公},\lycharname{桓公}(?--前643年):名小白,齐僖公三子,齐国第15任国君,春秋五\lylink{badao}{霸}之首(“辟土服远曰桓”)。在位43年,信任\lycharlink{guanzhong}{管仲}主政,力行改革,国力大盛,尊王攘\lylink{yidihuaxia}{夷},\lylink{jiuhezhuhou}{九合诸侯},成为“霸主”的典范。事见《史记·齐太公世家》。

\lylabel{qijinggong}
\lycharname{齐景公}(前547年--前490年在位):名杵臼,齐国第23任国君,\lycharlink{cuizhu}{齐庄公}的异母弟,在位之久为齐国之最。他\lyq{好治宫室,聚狗马,奢侈,厚赋重刑},幸好内有贤相\lycharlink{yanpingzhong}{晏婴},外有良将司马穰苴(\lypy{ráng jū}),任内局势基本稳定。但早在景公九年,晏婴就察觉到衰败的端倪,认为\lyq{齐政卒归\lycharlink{chenchengzi}{田氏}:田氏虽无大德,以公权私,有德于民,民爱之。}事见《史记·齐太公世家》。《晏子春秋》记载了大量“景公如何如何,晏子谏/对/讽”的故事。

\lylabel{quboyu}
\lycharname{蘧(\lypy{qú})伯玉}:蘧瑗(\lypy{yuàn}),字伯玉,卫国贤大夫,谥号是成(“安民立政曰成”)。他应比孔子年长10岁以上,孔子在卫国时,曾借住在他家。西汉刘安《淮南子·原道训》说:\lyq{蘧伯玉年五十而有四十九年非。何者?先者难为知,而后者易为攻也。}(攻:指借鉴经验以克服困难,相当于现代说的后发优势。)可见他是个不断追求进步的人。据《孔子家语·弟子行》,孔子评价他\lyq{外宽而内正,自极于隐括之中,直己而不直人,汲汲于仁,以善自终。}(隐括:矫正曲斜的工具,隐是矫曲为直,括是矫斜为方,引申为标准规范,构词与\lylink{leixie}{缧绁}相同。汲汲:急切追求的样子。)
% NOTE: 他书对《庄子》、《淮南子》的引用多不准确。年长10岁为推论,因《左传·襄公十四年》(-559)蘧伯玉已为官。TODO: Add link to 孔子在卫国。

\lylabel{shusunwushu}
\lycharname{叔孙武叔}:鲁国\lycharlink{sanhuan}{三桓}之一叔孙氏的第8代宗主,名州仇,谥号是武,曾任鲁国\lycharlink{huantui}{司马}。叔孙氏的封邑是郈(\lypy{hòu})邑,在今山东省泰安市东平县,位于\lylink{wenshui}{汶水}北岸,齐鲁交界处。

% NOTE: 郈邑之叛。起初,邑宰公若藐曾强烈反对把叔孙武叔立为继承人。前505年武叔继位,派人暗害公若藐未果。5年后(《左传·定公十年》),又指使郈邑的马正(相当于大夫家的\lycharlink{huantui}{司马})侯犯刺杀了公若藐。可侯犯趁机占据郈邑为叛,武叔联合\lycharlink{mengyizi}{孟懿子}与齐军久攻不克,后来串通郈邑的工师(工匠长官)驷赤使用反间计,才将侯犯逼往齐国,还差点把郈邑也送给齐国了。

\lylabel{shun}
\lycharname{舜}(约前2200年前后):有虞氏,名重华,又称虞舜,上古传说中\lycharlink{wudi}{五帝}的第5位,是五帝第2位颛顼的6世孙。他20岁即以孝谨闻名(成为《二十四孝》第1个故事“孝感动天”的主角),30岁被举荐给\lycharlink{yao}{尧帝}。尧帝把两个女儿娥皇、女英嫁给他,让9个儿子同他共事。经过长期的考察锻炼,舜在50岁时摄天子政,61岁登帝位,22年后退居二线,授权给治水有功的\lycharlink{yu}{禹},又17年后去世。舜帝\lylink{wuweierzhi}{举贤任能},布教四方,厚德远佞,建立了井井有条的清平世界。

\lylabel{wangsunjia}
\lycharname{王孙贾}:\lycharlink{weilinggong}{卫灵公}时的卫国大夫。

\lylabel{weijun}
\lycharname{卫君}:指当时在位的\lykw{卫出公}蒯辄(?--前456年),\lycharlink{weilinggong}{卫灵公}之孙。他的父亲蒯聩原被灵公立为太子,前496年,因谋杀\lycharlink{nanzi}{南子}不成被逐,投奔与卫国\lylink{liuqingneizhan}{结怨}的晋权臣赵简子。前493年灵公去世,蒯辄继位,赵简子派\lycharlink{yanghuo}{阳货}带队,送助蒯聩回卫争位,形成父子相斗的局面,\lyref{7.15} \lyq{夫子为卫君乎}的对话就发生在这期间。结果蒯聩被卫人阻截,无法入境。前480年,蒯聩由姐姐(\lycharlink{kongwenzi}{孔文子}的孀妻)做内应混入都城,挟持了外甥\lycharlink{kongkui}{孔悝}发动叛乱,终于夺位成为\lykw{卫庄公},蒯辄奔鲁,\lycharlink{zilu}{子路}为营救孔悝,死于这场变故。3年后,晋又围卫,蒯聩出逃后被杀,蒯辄回国复位。事见《史记·卫康叔世家》。% TODO: add link;重新统计年份 % NOTE: 前493年蒯聩是否入卫境,《左传》、《史记》有分歧,此依《史记》。

\lyq{夫子为卫君乎}时的孔子,是周游列国踌躇满志的政治家。卫庄公上台时的孔子,已经是风烛残年、桃李满天下的老人,一年后就去世了。

\lylabel{weilinggong}
\lycharname{卫灵公}(前540年--前493年):名元,\lylink{kangshu}{卫国}第28任国君,6岁继位(“乱而不损曰灵”)。他的私生活较混乱,内惧夫人\lycharlink{nanzi}{南子},\lylabel{mizixia}信任男宠弥子瑕(典故“断袖余桃”后一半的主角),但公务方面称得上知人善任。《孔子家语·贤君》载,\lycharlink{aigong}{鲁哀公}曾问孔子当今之君谁为最贤,孔子回答:尽管称不上贤,卫灵公也算佼佼者了:\lyq{臣语其朝廷行事,不论其私家之际也。}

\lylabel{wen}
\lycharname{文},\lycharname{文王}:周文王姬昌(前1152年--前1056年),商末周国首领,\lycharlink{taibo}{季历}之子,\lycharlink{wu}{周武王}之父,在位51年,被\lycharlink{zhou}{殷纣王}封为西伯,武王灭商后追封其为周文王。文王在位时,笃仁积善,敬老慈少,天下归心,受到纣王忌惮,被囚禁于羑(\lypy{yǒu})里(在今河南省安阳市汤阴县北)。据说期间将\lycharlink{sanhuang}{伏羲}发明的八卦两两相重,推演为六十四卦(又称后天八卦),形成了流传至今的《\lylink{yijing}{周易}》。囚禁近8年后,文王的谋臣散宜生、\lycharlink{jiangtaigong}{吕尚}等人献美女重宝于纣王,纣王大悦,不仅赦免了文王,还赐予他\lylink{liyuezhengfa}{征伐之权},周的势力\lylink{sanfentianxia}{日益扩大}了。文王回国约6年后去世,寿至97岁,武王继立。事见《史记·殷本纪》、《史记·周本纪》。% NOTE: 文王、武王的寿数,说法不一,此取自《礼记·文王世子》。

《诗经》有很多诗篇赞颂文王的美德。民间传说文王有百子,依据的只是《诗经·大雅·思齐》:\lyq{大姒嗣徽音,则百斯男。}(\lylabel{taisi}\lylink{tai4}{大}姒:即太姒,文王的正妻。徽:美,善。)文王的长子是伯邑考,《帝王世纪》载,文王遭囚后,他自愿入商做人质,被纣王杀害:\lyq{质于殷,为纣御。纣烹以为羹,赐文王,曰:“圣人当不食其子羹。”文王得而食之,纣曰:“谁谓西伯圣者?食其子羹尚不知也。”} % NOTE: 《大戴礼记》:\lyq{文王十三生伯邑考,十五而生武王},有出入。

\lylabel{wu}
\lycharname{武},\lycharname{武王}:周武王姬发(前1135年--前1043年),\lylink{zhouchao}{周朝}开国之君,\lycharlink{wen}{周文王}次子。前1056年继位,以姜太公\lycharlink{jiangtaigong}{吕尚}为师,任用\lycharlink{zhougong}{周公旦}、\lycharlink{shaogong}{召公奭}等贤臣,\lyq{师修文王\lylink{xu4b}{绪}业}。8年后,在盟津(今河南省洛阳市孟津县)阅兵扬威,\lyq{不期而会盟津者八百诸侯},且天降异兆,白鱼跃入王舟,天火化为赤鸦。诸侯都说\lyq{\lycharlink{zhou}{纣}可伐矣},武王认为\lyq{未知\lylink{tianming}{天命}}(应是未有把握),引兵而还。后来纣王杀\lycharlink{biganjiziweizi}{比干}、囚\lycharlink{biganjiziweizi}{箕子},众叛亲离,武王认为不可不伐,遂于前1046年春,率5万精兵,在\lylink{zhaoge}{朝歌}郊外的牧野(今河南省新乡市北)一日尽破号称70万的商军,纣王自焚死,商亡周兴。3年后武王去世,寿至93岁。事见《史记·周本纪》。
% NOTE: 5万精兵 = 戎车300乘 + 虎贲3000 + 甲士4.5万。70万《史记》如此,似有夸大。文王、武王的寿数说法颇多。

\lylabel{yanpingzhong}
\lycharname{晏平仲}:晏婴(前578年--前500年),平是他的谥号(“布纲治纪曰平”),仲是他的字。春秋后期齐国灵公、庄公、\lycharlink{qijinggong}{景公}的三朝重臣,娴于外交,智慧通达,节俭力行。《史记·管晏列传》简要记述了他的事迹,《晏子春秋》集中载录了他的言行活动。

\lylabel{yao}
\lycharname{尧}(约前2300年前后):陶唐氏,名放勋,又称唐尧,上古传说中\lycharlink{wudi}{五帝}的第4位,是五帝第3位喾的次子。尧20岁登帝位,仁智明德,亲亲利民,《史记·五帝本纪》形容他\lyq{其仁如天,其知如神,就之如日,望之如云。}他在位70年后退居二线,授权给有孝德的平民\lycharlink{shun}{舜},20年后荐舜摄天子政,又8年后去世,\lyq{百姓悲哀,如丧父母。}后来“尧舜\lycharlink{yu}{禹}\lylink{tang}{汤}”、“二帝三王”(尧、舜;禹、汤、\lycharlink{wu}{周武王})被儒家奉为圣明君主的代名词。

\lylabel{yegong}
\lycharname{叶(旧读\lypy{shè})公}:沈诸梁,字子高,楚国大臣,曾被楚昭王任命为叶县(在今河南省叶县南)长官,故称叶公,是叶姓始祖。在叶任内,他积极兴修水利,深受民众爱戴。前479年(同年孔子去世),楚惠王之孙谋反,叶公领兵平定了叛乱,这时他身兼令尹(最高政务长官)、司马(最高军事长官)二职,局势平静后却立即退休,终老于叶,享年89岁。传说为求龙王保佑风调雨顺,叶公让在每个出水口画上龙像,成为“水龙头”的由来。“叶公好龙”是西汉刘向《新序》假托叶公之名虚构的寓言。

\lylabel{yu}
\lycharname{禹}(约前2100年前后):夏后氏,名文命,又称夏禹、大禹。他是中国历史上最伟大的水利工程师,\lycharlink{yao}{尧帝}时受\lycharlink{shun}{舜}举荐,治理肆虐天下的\lylink{dayuzhishui}{洪水},\lyq{劳身焦思,居外十三年,过家门不敢入}。采用疏导之法,定\lylink{sanfentianxia}{九州}、测九山、通九道、筑九堤,终告成功,使\lylink{zhongyuan}{中原}成为宜居之地。禹为人勤勉务实,以身作则,可亲可信,受舜禅让为帝,在位10年后去世。
% NOTE: 《史记·夏本纪》说他是\lycharlink{wudi}{五帝}第2位颛顼的孙子,五帝第1位黄帝的玄孙,应有误。1943年,中华民国政府将公历6月6日(传说是禹的生日)定为工程师节,又称水利节。

\lylabel{boyi}
伯益,又称益,是禹指定的继承人,他是\lylink{yingzheng}{嬴姓}始祖,舜帝的山泽畜牧之官,曾助禹治水有功。但他佐政\lylink{gaoyao}{日浅},根基不牢,\lylabel{qi3}禹的儿子启又有贤能,所以当他表示谦让时,诸侯就顺势拥启为帝,开启了中国第一个世袭制王朝\lylink{xiachao}{夏朝}。
% NOTE: 《古本竹书纪年》的说法是:\lyq{益\lylink{gan1}{干}启位,启杀之。}

\lylabel{zangwenzhong}
\lycharname{臧(\lypy{zāng})文仲}(?--前617年):名辰,谥号是文,鲁国庄公、僖公、文公的三朝重臣,为卿50余年。他思想开明,重民利商,娴于外交,《左传》、《国语·鲁语上》均有记载。前666年鲁国闹饥荒,他自告奋勇赴齐国借粮,并向质疑者解释说:\lyq{“贤者急\lylink{bing4}{病}而让夷,居官者当事不避难,在位者恤民之患,是以国家无违。”}(夷:平坦无碍。急病而让夷相当于趋难让易。\lylabel{xu4c}恤:怜悯,体念。)臧氏家族的名望仅次于\lycharlink{sanhuan}{三桓},而且是鲁国唯一历经春秋始终的世卿大族。

前549年(《左传·襄公二十四年》),叔孙氏第5代宗主叔孙豹访晋,\lylabel{fanxuanzi}晋国正卿范宣子请教他什么叫“死而不朽”,他回答:\lyq{鲁有先大夫曰臧文仲,既\lylink{mo4a}{没},其言立,其是之谓乎?豹闻之:大上有立德,其次有立功,其次有立言,虽久不废,此之谓不朽。……禄之大者,不可谓不朽。}臧文仲最著名的话,也许就是前683年(《左传·庄公十一年》)评价\lylink{songguo}{宋国}当兴时说的:\lyq{\lylink{yu}{禹}\lycharlink{tang}{汤}罪己,其兴也悖焉;\lycharlink{xiachao}{桀}\lylink{zhou}{纣}罪人,其亡也忽焉。}(悖:通“勃”,兴盛的样子。)

《论语》中两次论及臧文仲都持否定态度。《左传·文公二年》(前625年)插入了孔子对他的集中性评论,可以看出孔子衡量为政者的所轻所重:\lyq{“臧文仲,其不仁者三,不知者三:\lycharlink{15.14}{下展禽},废六关,妾织蒲,三不仁也;\lylink{5.18}{作虚器},纵逆祀,\lylink{siyuanju}{祀爰居},三不知也。”}(废六关:指废除了六个收税关卡以利工商。织蒲:织蒲席卖钱,孔子认为不合礼法。\lylabel{zongnisi}纵逆祀:指该年秋臧文仲未能阻止鲁大夫夏父弗忌按\lyq{新鬼大,故鬼小}的顺序,把去世不久的\lycharlink{luxigong}{鲁僖公}排在前面祭祀。)

\lylabel{zangwuzhong}
\lycharname{臧武仲}:名纥(\lypy{hé}),谥号是武,\lycharlink{zangwenzhong}{臧文仲}的孙子,曾任鲁国大夫。起先臧武仲被立为臧氏继承人,当时孟孙氏宗主\lycharlink{mengzhuangzi}{孟庄子}讨厌他,而季孙氏宗主季武子喜欢他。前550年孟庄子去世,臧武仲吊唁时哭得非常动情,认为孟庄子像良药苦口,季武子像美疢多毒(疢(\lypy{chèn}):热病),预感自己很快就会遭殃。果然几个月后,季武子就听信挑拨认为臧武仲要作乱,将他驱逐出鲁国。臧武仲先逃到\lylink{zhuguo}{邾国},又流亡至齐,\lycharlink{cuizhu}{齐庄公}准备赏给他封地。臧武仲也许有感于庄公平时的言行,当面讽刺庄公此前趁晋国内乱攻城略地的行为像老鼠一样无耻,就未被封赏,算是明哲保身。《左传·襄公二十三年》末尾,孔子感慨\lyq{知之难也},认为以臧武仲之智仍不容于祖国,并非无缘无故,而是他的某些行为不合常理,有失恕道。

\lylabel{zhougong}
\lycharname{周公}:姬旦,\lycharlink{wen}{周文王}四子,周初文、武、成王的三朝重臣,谥号是文(“经纬天地曰文”)。周朝建立后,他被\lycharlink{wu}{武王}封于鲁(都城在今山东省曲阜市),由长子\lycharlink{lugong}{伯禽}代任,自己一直辅佐周王室,在\lylink{wangji}{王畿}内另有\lylink{jia1}{采邑}周城,故称周公。

周公旦具有高尚的品德,卓越的才能,是孔子倾心仰慕的大\lylink{sheng4}{圣人}。\lyq{自文王在时,旦为子孝,笃仁,异于群子。及武王即位,旦常辅翼武王,用事居多。}武王有疾,周公设坛祭祖,情愿以身相代。\lylabel{chengwang}武王去世后,继位的\lykw{成王}(姬诵,前1042年--前1021年在位)年少,周公摄政当国,威仪如君;7年后还政成王,恭敬称臣。成王患病,周公又祷神自代。他和\lycharlink{shaogong}{召公}主持营建的新城洛邑,成为后来\lylink{zhouchao}{东周}的都城;他参照\lylink{shangchao}{殷商}的传统礼仪,制定了周朝的礼乐制度,相传是儒家经典《周礼》的作者;他还多次平定了诸侯叛乱,使成王及随后的\lykw{康王}(姬钊,前1020年--前996年在位)任内,\lyq{天下安宁,刑\lylink{cuo4}{错}四十余年不用},后来\lykw{文武成康}就成为盛世的代名词。《礼记·明堂位》记载:\lyq{成王以周公为有勋劳于天下,是以封周公于曲阜,地方七百里,革车千乘,命鲁公世世祀周公以天子之礼乐},鲁国得以保留了周王室的大量文献制度,并受到其它诸侯国的尊重。事见《史记·周本纪》、《史记·鲁周公世家》、《尚书·周书》。
% NOTE: 成王继位之年,有6、10、13诸说不一,据摄政时间,似应为13岁。别国以鲁之礼制为尊,例见《孟子·滕文公上》:\lyq{吾宗国鲁先君莫之行。}

\lylabel{shaogong}
召(\lypy{shào})公奭(\lypy{shì}):姬奭,周初文、武、成、康的四朝重臣,谥号是康(“安乐抚民曰康”)。周朝建立后,他被\lycharlink{wu}{武王}封于燕(\lypy{yān},都城在今北京市房山区),由长子克代任,自己一直辅佐周王室,在王畿内另有采邑召城,故称召公。《韩诗外传》第1卷第28章说,召公辅政时,不忍\lyq{以吾一身而劳百姓},拒绝营造豪华的办公楼,而是亲赴乡野田间听取民众诉求,止歇在甘棠树下。后人睹物思人,写下了《诗经·召南·甘棠》:\lyq{蔽芾甘棠,勿剪勿败!召伯所憩。}(甘棠:又名棠梨、豆梨,果实圆形褐色,小如樱桃。蔽芾(\lypy{fèi}):枝叶茂盛浓郁的样子。)

周召二公的采邑是从\lycharlink{taibo}{古公亶父}创立的旧周国划分的。周公主抓国政,召公主抓外交,成王年少时,二公代理天下:\lyq{自陕以西,召公主之;自陕以东,周公主之。}(陕:一说为今河南省三门峡市陕州区,这样分则周公负责的范围大得多;一说为王城郏鄏(\lypy{jiá rǔ}),在今河南省洛阳市内,这样分则较均衡。)召公与周公既密切配合又互有制约,成就了西周的鼎盛时代。事见《史记·燕召公世家》、《尚书·周书》。

周公旦、召公奭去世后,周公、召公就成为他们后代的专有封号,继任者同样留在周廷辅政。% TODO: Verify 官职, sync with 3.1.

\lylabel{zhutuo}
\lycharname{祝鮀}(\lypy{tuó}):卫国太祝(负责宗庙祭祀),名鮀,又称祝佗,字子鱼,以口才著称。前506年春(《左传·定公四年》),近20位诸侯首脑举行集会,商议联合攻打楚国。卫国大臣认为这样的大会难得意见一致,争执起来很麻烦,所以强请祝鮀与\lycharlink{weilinggong}{卫灵公}同行。会议还没开始,灵公就对蔡国(武王五弟蔡叔的封国)位居卫国(武王九弟康叔的封国)之前产生不满。于是祝鮀引述先王之法,提出\lylink{shang4a}{尚}德不尚年(各国始祖的长幼关系),发表了滔滔雄辩,为祖国争得了靠前的位次。明代刘基的寓言《郁离子·\lycharlink{mizixia}{弥子瑕}》中,还借祝鮀之口对仰人鼻息的奴才走狗进行了尖锐讽刺。

\lylabel{zichan}
\lycharname{子产}(?--前522年):国侨,字子产,又称公孙侨,谥号是成(“安民立政曰成”)。辅佐郑简公、郑成公,为郑相26年,以仁智著称,内惠于民,外善辞令,使郑国安存于晋楚两个大国之间。《史记·循吏列传》载,子产去世时,\lyq{丁壮号哭,老人儿啼,曰:“子产去我死乎!民将安归?”}《左传·昭公二十年》载,孔子也盛赞子产之政,\lyq{及子产卒,仲尼闻之,出涕曰:“古之遗爱也!”}(古之遗爱:像古仁人那样值得怀念爱戴的品德。)
% NOTE: 《史记·郑世家》载,\lyq{孔子尝过郑,与子产如兄弟云},应有误。

子产的事迹在《史记·郑世家》、《左传》襄公八年(前565年)至昭公二十年(前522年)多有记述,孔子之后历代皆有赞颂和评价。\lylabel{zhuxingding}前536年,他一改\lyq{刑不可知,则威不可测}的陈规(《左传·昭公六年》孔颖达疏),将郑国刑法铸于鼎上公布,成为有据可依的铁律,是中国第一部成文法。\lylabel{buhuixiangxiao}《左传·襄公三十一年》(前542年)载,郑国人业余常聚在乡校(乡间平民学校,兼有广场、公园的性质)议论国政是非,有官员\lyq{谓子产曰:“毁乡校,何如?”子产曰:“何为?夫人朝夕退而游焉,以议执政之善否。其所善者,吾则行之;其所恶者,吾则改之。是吾师也,若之何毁之?我闻忠善以损怨,不闻作威以防怨。岂不遽止,然犹防川:大决所犯,伤人必多,吾不克救也;不如小决使道,不如吾闻而药之也。”}(\lylabel{ju4}遽(\lypy{jù}):匆忙,急迫。决:开挖,疏通。药:动词,治疗。)

清代姜炳璋《读左补义》认为:\lyq{《春秋》上半部,得一\lycharlink{guanzhong}{管仲},《春秋》下半部,得一子产,都是救时之相。管仲之功阔大,泽在天下,然其过多。子产之才精实,功在一国,然其过少。管仲死而\lylink{qiluan}{齐乱},以贤才不用,而小人得志也。子产死而郑治,以犹用子太叔也。}(\lylabel{zitaishu}子太叔:郑国大臣,曾辅佐子产,又继子产为郑相。)

\lylabel{zifujingbo}
\lycharname{子服景伯}:子服何,谥号是景,\lycharlink{aigong}{鲁哀公}时期的鲁国大夫,子服氏宗主。他和\lycharlink{zigong}{子贡}应该是关系不错的同事,吴、卫发生纠纷时,他请子贡出面斡旋,后来出使吴国也带子贡为副使,事见《左传》哀公十二年、十五年。

