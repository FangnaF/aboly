\lylabel{references}
\phantomsection
\chapter*{参\quad 考\quad 材\quad 料}
\addcontentsline{toc}{chapter}{参考材料}
%%%%%%%%%%%%%%%%%%%%%%%%%%%%%%%%%%%%%%%%%%%%%%%%%%%%%%%%%%%%

本书引用的内容,只是依据笔者的眼界好恶“择善而从”,偏颇在所难免。这里列出了主要的参考材料,以便读者查阅求证。其它材料(尤其是网上资料)无法详列,这里一并谨致谢忱。

列出的每本书后面,有的带有点评,很多还没来得及加上。主要是背景介绍和很浅陋的一点感受,不敢对前辈的心血不敬,请读者明察。

\lylabel{baiwenban}
岳麓书社“古典名著普及文库”收录的都是“白文版”,也就是无注释无翻译不删节的古文原本。因为没有视觉上的不连续性,就易于把注意力集中到原文上,形成完整独立的认识。后面不逐一评介。
% NOTE: 可惜大多把古人序跋删掉了,有上下文缺失之憾。中华书局的“中华国学文库”这方面就不错,但“中华经典普及文库”、“中华经典名著全本全注全译丛书”也有同样的问题。


\lypdfbookmark\section*{工具书}
%%%%%%%%%%%%%%%%%%%%%%%%%%%%%%%%%%%%%%%%%%%%%%%%%%%%%%%%%%%%

\lybook{《辞源》}(修订本),商务印书馆编辑部\,等编,商务印书馆,1988年。学习古汉语的权威参考,注解扎实,例句常引自最初的用法,有一锤定音的效果。但从检索到印刷,随处可以察觉它的高龄,释义、引文偶有不确。对比朗文、韦氏等英文词典的勤修苦练子孙满堂,可改进的地方实在太多,愿关心\lylink{lunyujishi}{国家生命}者察焉。2015年底出了“有限修订”的第3版。
% NOTE: 引文不确的例子,如“艺”的子罕篇例句。第3版网络版(http://ciyuan.cp.com.cn/)需要付费,目前(2016-2)感觉还有bug,体验欠佳。“有限修订”出自第三版前言。另外,前言“太初有字,道在其中,理在其中”等提法,感觉很奇怪。可以和1915年初版前言“《辞源》说略”的精气神作比较(http://ciyuan.cp.com.cn/etymology/common/showHTML.jspx?url=/fulu/chubanshuoming.html)。

\lybook{《古代汉语词典》}(第2版),张双棣\,主编,商务印书馆,2014年。主要用它和《辞源》相互印证。常比较不同的词典把哪个义项放在最前面作为本义(第一推动),可以揣摩编撰者进退取舍的思路。
% NOTE: 编校略粗疏,如“第2版修订说明”中,“楚辞”赫然误为“楚语”。且无附录。

\lybook{《古汉语常用字字典》}(第4版),王力\,主编,商务印书馆,2005年。这是我最熟悉亲切的古汉语字典,详略合宜,释义清晰,例证精当,上中学时得益甚多。本书常举成语为例、不怕多注拼音,可能就是受它的影响。

\lybook{《古代汉语虚词词典》},中国社会科学院语言研究所古代汉语研究室\,编,商务印书馆,1999年。包括单字和常用搭配,共1855条,讲解细密,不落俗套,引证丰富,令人心情爽快。

\lybook{《汉字源流字典》},谷衍奎\,编,语文出版社,2008年。上推至甲骨文、篆文,分析了11000多个汉字的构造、本义与演变,图文并茂,多有创见,特别有益于理清字义的发展脉络。

\lybook{《中国历史地图集》(第1册:原始社会·商·西周·春秋·战国时期)},谭其骧\,主编,中国地图出版社,1982年。这是一套划时代的地图集,编制历时近30年,从谭先生留下的日记中,不难感受当时的工作热忱。30多年又过去了,最畅销的还是这个版本,当年为了“尽快公开出版”而留下的“很不少”的“错误和不妥之处”,仍待后人。
% NOTE: 谭先生日记及回忆,见葛剑雄编《谭其骧日记》,著《悠悠长水:谭其骧传》。地图集“出版后记”中写道:\lyq{我可以在此奉告读者:从1982年起,中国社会科学院已经组织了有关的学者专家着手编制一部包括各种题材的巨型历史地图,争取在九十年代问世。}


\lypdfbookmark\section*{论语释译}
%%%%%%%%%%%%%%%%%%%%%%%%%%%%%%%%%%%%%%%%%%%%%%%%%%%%%%%%%%%%

\lylabel{lunyuyizhu}\lybook{《论语译注》}(第2版),杨伯峻\,译注,中华书局,1980年。这应该是解放后中国大陆最流行的《论语》读本,注释翔实,脉络分明,还有很多对古汉语的独到体会。限于时代,未能很好地阐述每一章的总体意义,也有一些小错。2006年中华书局出了简体字本,2009年岳麓书社出了杨先生堂侄杨逢彬负责的修订重版。奇怪的是,它们把原本颇具特色的“试论孔子”、“论语词典”等删掉,只留下光溜溜的20篇。我觉得,往往是这样推陈出新的地方,才体现作者的功力,才对读者更有用。幸好2015年中华书局典藏版复其原貌。
% NOTE: 错误辨析可以参考2012年白平的《杨伯峻〈论语译注〉商榷》。其中对译文的一些纠结,似乎不太必要。

\lybook{《论语新解》},钱穆\,著,三联书店,钱穆作品系列,2002年。以浅近的文言文写成,字词的注释较简明,着重于“心解”每一章的意义,反复阐述修身治学之道,在现代注本中占有不可替代的地位。

\lybook{《论语本解》},孙钦善\,著,三联书店,2009年。注释精准,各章之间经常提供交叉引用,5个附论也各有价值。2013年三联书店出了修订版。

\lybook{《论语今注今译》},毛子水\,注译,重庆出版社,2009年。原属台湾商务印书馆“古籍今注今译丛书”的一种,1975年出版,1984年作者校订再版,在海外具有广泛持久的影响。它的大部分注释选自合适的古注,辅以按语,另有不少生动活泼的议论,遇到难理解或争议大的就直接写明,读起来既顺畅又踏实。该丛书由上海商务印书馆前总经理(1930--1945)、台湾商务印书馆前董事长(1964--1979)王云五先生主编,共出版约60种。

\lybook{《论语通译》}(修订版),徐志刚\,译注,人民文学出版社,语文新课标必读丛书,2006年。发行量较大,简明清通,但似乎存在过于简化的倾向,也不够细致,从注到译疏漏略多。
% NOTE: 例如:学则不固;三年无改于父之道。

\lybook{《论语注疏》},〔魏〕何晏\,集解,〔梁〕皇侃\,义疏,〔北宋〕邢昺\,正义,梁运华\,整理,山东画报出版社,四库家藏丛书,2004年。

\lybook{《四书集注》},〔南宋〕朱熹\,集注,陈戍国\,标点,岳麓书社,2004年。

\lylabel{lunyujishi}
\lybook{《论语集释》},程树德\,撰,中华书局,中华国学文库,2013年。本书可称《论语》注解大全,分类集成了从汉至清各家学者的研究心得,引书680种,并加按语,方便后人系统了解各种诠释与引申,至可感佩。或有讥为取舍不尽精当,敢问更精者何在?其《自序》称:\lyq{夫文化者国家之生命,思想者人民之倾向,教育者立国之根本,凡爱其国者,未有不爱其国之文化。}可与钱穆先生《国史大纲》的开篇“凡读本书请先具下列诸信念”对读。

\lybook{《唐写本论语郑氏注及其研究》},王素\,编著,文物出版社,1991年。

\lybook{《孔子集语校补》},〔清〕孙星衍\,等辑,郭沂\,校补,齐鲁书社,1998年。

\lybooke{Analects: With Selections from Traditional Commentaries}, Edward Slingerland(森舸澜), Hackett Publishing Co., Hackett Classics, 2010.
% NOTE: 解释不确的例子,如“天将以夫子为木铎”,被翻译成 "Heaven intends to use your Master like the wooden clapper for a bell."

\lybooke{CONFUCIUS: Discussions/Conversations, or the Analects (Lun-yu)}, David R. Schiller, Amazon Digital Services, Inc., 2014.


\lypdfbookmark\section*{孔子生平}
%%%%%%%%%%%%%%%%%%%%%%%%%%%%%%%%%%%%%%%%%%%%%%%%%%%%%%%%%%%%

\lybook{《孔子》},张秉楠\,著,人民日报出版社,2010年。

\lybook{《孔子传》},钱穆\,著,三联书店,钱穆作品系列,2012年。

\lybook{《先秦诸子系年(外一种)》},钱穆\,著,河北教育出版社,二十世纪中国史学名著,2001年。

\lybook{《孔子传》},李长之\,著,东方出版社,2010年。

\lybook{《论语二十讲》},周予同、朱维铮\,等著,傅杰\,选编,华夏出版社,2009年。

\lybook{《孔子与中国之道》}(\lybooke{Confucius and the Chinese Way}),顾立雅(Herrlee Glessner Creel)著,高专诚\,译,大象出版社,当代海外汉学名著译丛,2000年。Original English edition published by Harper Torchbooks, 1960.

\lybook{《孔子:喧嚣时代的孤独哲人》}(\lybooke{The Authentic Confucius: A Life of Thought and Politics}),金安平\,著,黄煜文\,译,广西师范大学出版社,2011年。Original English edition published by Scribner, 2007.

% \lybooke{Confucius: And the World He Created}, Michael Schuman, Basic Books, 2015.


\lypdfbookmark\section*{引文参考}
%%%%%%%%%%%%%%%%%%%%%%%%%%%%%%%%%%%%%%%%%%%%%%%%%%%%%%%%%%%%

\lybook{《史记》},李全华\,标点,岳麓书社,古典名著普及文库,1988年。

\lybook{《史记笺证》}(修订版),韩兆琦\,编著,江西人民版,2009年。

\lybook{《汉书》},陈焕良、曾宪礼\,标点,岳麓书社,2008年。

\lybook{《后汉书》},陈焕良、李传书\,标点,岳麓书社,1993年。

\lybook{《四书五经》},陈戍国\,点校,岳麓书社,2003年。白文版,11合1(含春秋三传),使用方便。
% NOTE: 此为上下册1637页。另有中华书局“中华经典普及文库”同名书,单册925页,不含公羊、谷梁传。

\lybook{《春秋左传正义》},〔西晋〕杜预\,注,〔唐〕孔颖达\,正义,北京大学出版社,十三经注疏丛书,1999年。

\lybook{《左传译注》},李梦生\,撰,上海古籍出版社,中华古籍译注丛书,1998年。

\lybook{《尚书译注》},李民、王健\,撰,上海古籍出版社,国学经典译注丛书,2012年。《尚书》篇幅不长,但以诘屈聱牙著称。这个译注本精细周详,有历史背景分析,有各家注的比较,引证充分,点评得当,给人呵护备至的感觉。

\lylabel{shijingquanzhu}\lybook{《诗经全注》},褚斌杰\,注,人民文学出版社,世界文学名著文库,1999年。

\lybook{《诗经校注》},陈戍国\,撰,岳麓书社,古典名著标准读本,2004年。

\lybook{《诗集传》},〔南宋〕朱熹\,注,岳麓书社,古典名著普及文库,1989年。

\lybook{《周礼·仪礼·礼记》},陈戍国\,点校,岳麓书社,古典名著普及文库,2006年。

\lybook{《周礼》},徐正英、常佩雨\,译注,中华书局,中华经典名著全本全注全译丛书,2014年。

\lybook{《礼记译注》},杨天宇\,撰,上海古籍出版社,十三经译注丛书,2004年。

\lybook{《三礼辞典》},钱玄、钱兴奇\,编著,江苏古籍出版社,1998年。

\lybook{《大戴礼记今注今译》},高明\,注译,台湾商务印书馆,古籍今注今译丛书,1975年。

\lybook{《周易译注》},黄寿祺、张善文\,撰,上海古籍出版社,十三经译注丛书,2004年。

\lybook{《周易正义》},〔魏〕王弼、〔东晋〕韩伯\,注,〔唐〕孔颖达\,等疏,何锡光、虎维铎\,整理,山东画报出版社,四库家藏丛书,2004年。
% NOTE: 王弼《道德经注》长期以来是《老子》流传于世的唯一版本,直到1973年从湖南省长沙市马王堆汉墓出土西汉初年的帛书本,1993年从湖北省荆门市出土战国中期的摘抄本(郭店楚简本)等。韩伯字康伯。

\lybook{《孝经·二十四孝》},喻岳衡、喻涵\,译注,岳麓书社,古典名著阅读无障碍本,2012年。

\lybook{《国语·战国策》},李维琦\,点校,岳麓书社,古典名著普及文库。2006年。

\lybook{《孔子家语》},〔魏〕王肃\,编著,王国轩、王秀梅\,译注,中华书局,中华经典名著全本全注全译丛书,2011年。

\lybook{《孟子译注》},杨伯峻\,编著,中华书局,1960年。

\lybook{《韩诗外传译注》},〔西汉〕韩婴\,编著,魏达纯\,译注,东北师范大学出版社,1993年。

\lybook{《韩诗外传今注今译》},赖炎元\,注译,台湾商务印书馆,古籍今注今译丛书,1972年。

\lybook{《老子·庄子·列子》},张震\,点校,岳麓书社,古典名著普及文库,2006年。

\lylabel{laozizhuyi}\lybook{《老子注译》},高亨\,著,华钟彦\,校,河南人民出版社,1980年。《老子》版本繁多,文字、断句屡有分歧,难辨原委,本书引文均据此本。

\lybook{《庄子》},方勇\,译注,中华书局,中华经典名著全本全注全译丛书,2010年。

\lybook{《管子通解》},赵守正\,撰,北京经济学院出版社,1989年。

\lybook{《吕氏春秋·淮南子》},杨坚\,点校,岳麓书社,古典名著普及文库,2006年。

\lybook{《商君书·韩非子》},张觉\,点校,岳麓书社,古典名著普及文库,2006年。附录含《申子》、《慎子》。

\lybook{《韩非子校注》},张觉\,校注,岳麓书社,古典名著标准读本,2006年。

\lybook{《韩非子校注》}(修订本),《韩非子》校注组\,编写,周勋初\,修订,凤凰出版社,2009年。

\lybook{《说苑译注》},〔西汉〕刘向\,撰,程翔\,译注,北京大学出版社,2009年。

\lybook{《帝王世纪·世本·逸周书·古本竹书纪年》},陆吉\,等校点,齐鲁书社,2010年。

\lybook{《逸周书校补注译》}(修订本),黄怀信\,著,三秦出版社,2006年。

\lybook{《古本竹书纪年译注》},李民\,等译注,中州古籍出版社,1990年。

\lybook{《日知录集释》},〔清〕顾炎武\,著,〔清〕黄汝成\,集释,栾保群、吕宗力\,校点,花山文艺出版社,1990年。

\lybook{《中国哲学文献选编》},陈荣捷\,编著,江苏教育出版社,国学书库·哲学类丛,2006年。

\lylabel{yaojijietijiqidufa}\lybook{《要籍解题及其读法》},梁启超\,著,岳麓书社,民国学术文化名著丛书,2010年。1923年,清华大学\lyq{指定十来部有永久价值的古书令学生们每学期选读一部或两部,想令他们得些国学常识而且养成自动的读书能力}。本书就是梁启超先生(1873--1929)在清华介绍这些古典要籍的讲义。1925年初版,共8讲,主要涉及11本书,《论语》冠首。它写于传统学术遭遇西学冲击引发的“国学热”中,当时梁先生没有原书可查,且时间匆忙,算是闭卷考试。但以他中西交融的深厚学养,写来毫不费力,偶有离题发挥之处,正可一窥他的读书兴味与昂扬精神。
% NOTE: 另有两个附录:1923年4、5月间,为《清华周刊》撰写的《国学入门书要目及其读法》,以及大致同时的《评胡适之的“一个最低限度的国学书目”》。所述种种常识,似乎\lylink{15.26}{今亡矣夫}!如开篇讲四书实为基本读物:\lyq{六七百年来,数岁孩童入三家村塾者,莫不以四书为主要读本。其书遂形成一般常识之基础,且为国民心理之总关键。}

\lybook{《春秋史》},顾德融、朱顺龙\,著,上海人民出版社,2001年。

\lybook{《战国史》}(增订本),杨宽\,著,上海人民出版社,2003年。

\lybook{《鲁国史》},郭克煜\,等著,人民出版社,1994年。

\lybook{《中国文化史》},柳诒徵\,编著,中国大百科全书出版社,中国学术丛书,1988年。

\lybook{《中国古代官制》},柏铮\,编,北京大学出版社,1989年。

\lybook{《中国历代官制简表》},卫文选\,著,山西人民出版社,1987年。
